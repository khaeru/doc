\documentclass[12pt,aspectratio=169]{beamer}
\usetheme{iiasa}

\usepackage{ifthen}
\ifthenelse{\equal{\detokenize{notes}}{\jobname}}{%
\setbeameroption{show only notes}%
}{
%
}

\usepackage[
  maxnames = 1,
  style = authoryear,
  giveninits,
  terseinits,
  maxcitenames = 3,
  ]{biblatex}
\addbibresource{all.bib}
% Small font in bibliography
\renewcommand*\bibfont{\small}
\addbibresource{all.bib}

\usepackage{minted}
\usepackage{pdfpages}

\usepackage{tikz}
\usetikzlibrary{calc}

\title{The iTEM Open Data process}
\subtitle{Goals, concepts, and experiences}
\institute{$^1$ Energy, Climate, and Environment (ECE) Program, IIASA}

\date{\texorpdfstring{GIZ workshop on transport emissions data\\
  Monday, 16 May 2022 — Leipzig, DE}%
  {2022-05-16}}

\author{\texorpdfstring{Dr. Paul Natsuo Kishimoto \scriptsize\newline
  \href{mailto:paul.kishimoto@iiasa.ac.at}%
       {\ttfamily <paul.kishimoto@iiasa.ac.at>}\thanks{}}%
  {Dr. Paul Natsuo Kishimoto <paul.kishimoto@iiasa.ac.at>}}

\begin{document}

\maketitle

\section{Background info}

\begin{frame}
\frametitle{\href{https://transportenergy.org}{\bfseries transportenergy.org}}

\framesubtitle{iTEM = International Transport Energy Modeling}

\begin{itemize}
  \item Informal group of \structure{modeling} practitioners: researchers in academia, NGOs, IGOs, staff, energy firms —producing quantitative projections to inform policy advocacy and decisions; other research.
  \item Since 2014, functions as a meeting place / venue for talking shop, comparing notes.
\end{itemize}

\medskip
Core organizing group/key instigators…

\smallskip
\begin{columns}[T]
\column{0.47\textwidth}
Sonia Yeh (Chalmers U.)

Lew Fulton (ITS-Davis)

David McCollum (IIASA → EPRI → U.S. DoE ORNL)

\column{0.47\textwidth}
G. Page Kyle (U.S. DoE PNNL)

Josh Miller (ICCT)

Paul Natsuo Kishimoto (MIT → IIASA)

\end{columns}

\smallskip
…with many close friends, including at ITF-OECD and IEA and in this room.

\end{frame}

\begin{frame}
\frametitle{Why iTEM Open Data?}

Outputs of global- / national-scope transport-energy models (G-/NTEMs)%
\footnote{includes broader-scope models in which transport is represented in\\\hspace{25mm} detail, e.g. integrated assessment models (IAMs).}
is \structure{very difficult to compare}:
\begin{enumerate}
  \item Different methods → iTEM's original purpose.
  \item Historical baselines/calibrations are not aligned for many basic transport measures.
  \item Wide variety of input data sources/assumptions/pre-processing required to represent entire transport systems.
\end{enumerate}
\smallskip
This confounds knowledge generation: “Are differences between Model A/B due to bad data, or genuine uncertainty about future outcomes?”

\medskip
In turn, why do (2) and (3) occur? → The \structure{work of data-gathering and preparation} is mostly done \textbf{individually} by modeling teams, in the course of model-building and use.

\end{frame}

\section{Concepts for open data}

\begin{frame}
\frametitle{Goals: processes, not products}

Full description: \href{https://transportenergy.org/open-data\#goals}{\bfseries transportenergy.org/open-data\#goals}

\bigskip
Avoid an enormous amount of duplicated work / mistakes.
\begin{itemize}
  \item You \emph{could} gather, inspect, clean, harmonize all the data yourself
  \item …but this is rarely the best use of your (staff's) time.
\end{itemize}

\bigskip
Emulate best practices in other disciplines, e.g.
\begin{description}
  \item [Finance / econ] Exchange rates, stock and commodity prices.
  \item [Earth science] Weather observations and forecasts; climate model input/output data.
\end{description}

\smallskip
…that have unlocked more sophisticated modeling and analysis, and thus more useful knowledge.

\end{frame}

\begin{frame}
\frametitle{Concepts for open data}

Some high-level ideas that inform the iTEM Open Data approach:

\bigskip
\Large
\begin{enumerate}
  \item Open—warts and all
  \item Common but decentralized
  \item Automated
  \item Flexible/configurable
\end{enumerate}

\end{frame}

\begin{frame}
\frametitle{1. Open—warts and all}

A lot of research claims to be “open” but is actually not reproducible:
\begin{enumerate}
  \item “We use data sets X, Y, and Z” → good, if these are findable, accessible, open.
  \item “We clean the data and prepare model inputs” …
  \item “We our (open) model to produce these outputs.” → good.
\end{enumerate}

\medskip
(2) is not a mere chore, but a core step in the research method!
If this step is not clearly described, the overall process cannot be replicated, validated, etc. independently → not science.

\bigskip
iTEM Open Data uses established practices of free/scientific software development for data handling code: \href{https://github.com/transportenergy/database}{\bfseries github.com/transportenergy/database}
\end{frame}

\begin{frame}
\frametitle{2. Common but decentralized}

iTEM does not want to be \emph{the sole or authoritative source} for all users' data needs.
\begin{itemize}
  \item We can't; no one can; to claim otherwise is only marketing, and usually a misallocation of resources.
  \item Data will always come from and be channeled to many places.
\end{itemize}

\bigskip
What \emph{can} be pooled as a common good: \structure{knowledge \emph{about} the data}: its structure, meaning, errata, and proper usage.

\begin{itemize}
  \item iTEM Open Data focuses on providing infrastructure (common tools \& workflows) so that data cleaning \& harmonization methods can be collected, improved by the user community, and reused by anyone for any purpose.
\end{itemize}

\end{frame}

\begin{frame}
\frametitle{3. Automated}

Remove manual labour from frequently repeated steps, e.g.:
\begin{itemize}
  \item Translate data from one format to another.
  \item Replace or ‘map’ one set of labels/definitions to another.
  \item (Dis)aggregate.
\end{itemize}

\bigskip
The steps become nearly costless, can be repeated exactly, easily, and frequently.

\begin{itemize}
  \item E.g. one upstream data source receives small updates → regenerate the entire harmonized data set with the push of a button (or even automatically).
  \item Corrections or changes to the harmonization methods → same.
\end{itemize}

\end{frame}

\begin{frame}
\frametitle{4. Flexible and configurable}

Users (modelers, others) will legitimately want to use different data for the same measures:
\begin{itemize}
  \item Different upstream sources.
  \item Different choices/assumptions in harmonization.
\end{itemize}

\bigskip
…so, let them!
\begin{itemize}
  \item Provide a “standard offering” using default settings for the automated workflow.
  \item Allow users to choose different settings and get the output they want.
  \item These products become more easily traceable and communicable.
    \begin{quote}
      “We used the iTEM data code version X.Y, with this settings file: …”
    \end{quote}
\end{itemize}

\end{frame}

\begin{frame}
\frametitle{Experiences}

\begin{description}
  \item [Skill sets.] Building automated data pipelines etc. requires skills that are abundant in general, but not in our discipline.

    → Recruit and support the right personnel (sitting wherever).

  \item [Standards.] Look for them and use them before inventing something new.

    → e.g. \href{https://sdmx.org/?page_id=2555/}{SDMX}, \href{https://en.wikipedia.org/wiki/ISO_3166-2}{ISO 3166-2}, \href{https://www.bipm.org/en/committees/jc/jcgm/publications}{VIM}, etc.

  \item [Culture.] Community consensus around building \& using public goods, replacing a paradigm of individualism.

    → This deck.

  \item [Resources.] Still in search of sustainable sources for funding or time commitments for core infrastructure.
\end{description}

\end{frame}

\end{document}
