\documentclass[12pt,aspectratio=169]{beamer}
\usetheme{iiasa}

\usepackage{ifthen}
\ifthenelse{\equal{\detokenize{notes}}{\jobname}}{%
\setbeameroption{show only notes}%
}{
%
}

\usepackage[
  maxnames = 1,
  style = authoryear,
  giveninits,
  terseinits,
  maxcitenames = 3,
  ]{biblatex}
\addbibresource{all.bib}
% Small font in bibliography
\renewcommand*\bibfont{\small}
\addbibresource{all.bib}

\usepackage{minted}
\usepackage{pdfpages}

\usepackage{tikz}
\usetikzlibrary{calc}

\title{The MESSAGE\emph{ix} software ecosystem}
\subtitle{A brief introduction}
\institute{Energy, Climate, and Environment (ECE) Program \\
  International Institute for Applied Systems Analysis (IIASA)}

\date{\texorpdfstring{Message Community Meeting\\
  Monday 09–10 May 2022}%
  {2022-05-09–10}}

\author{\texorpdfstring{Dr. Paul Natsuo Kishimoto \scriptsize\newline
  \href{mailto:paul.kishimoto@iiasa.ac.at}%
       {\ttfamily <paul.kishimoto@iiasa.ac.at>}}%
  {Dr. Paul Natsuo Kishimoto <paul.kishimoto@iiasa.ac.at>}}

\begin{document}

\maketitle

\begin{frame}
\frametitle{Model-based research}

…includes \structure{“modeling”} but also other activities: shaping research questions (RQs), literature review, writing, communicating results, etc.

\bigskip
“Modeling” itself can be broken down into many tasks:
\begin{itemize}
  \item Methodology: choose how to represent real-world phenomena in a (quantitative) model, to address RQs.
  \item Identify data needs \& available data; collect \& prepare data.
  \item Quantify common and alternate assumptions for scenarios.
  \item Write code to implement the model; iterate on the representation.
  \item Inspect results; produce tables, figures, calculate statistics, etc.
\end{itemize}

\end{frame}

\begin{frame}
\frametitle{MESSAGE\emph{ix} \& related software}
\framesubtitle{Core goals of the MESSAGE\emph{ix} team within IIASA/ECE}

Provide tools for energy systems and integrated assessment model (IAM)-based research that are:

\medskip
\begin{itemize}
  \item ready to implement the latest research methods in these domains.
  \item free, accessible, and \emph{genuinely} interoperable and reusable (\structure{FAIR}).
  \item \structure{reliable}: stable, tested, valid; caveats handled transparently.
  \item well-\structure{documented} and supported by learning materials and activities.
  \item \structure{developed} through best practices and principles from professional/ open/Free software development.
\end{itemize}

\medskip
…for use by ourselves \emph{and} a broader community of researchers— you!

\end{frame}

\begin{frame}
\frametitle{A substantial \& ongoing investment}

IIASA ECE's MESSAGE\emph{ix} team is $\sim 20$ people, most of whom spend $\sim 80\%$ time \emph{using} the software to do research.

\begin{itemize}
  \item New features developed as needed to enable/accelerate that research.
  \begin{itemize}
    \item Incremental improvement.
    \item High standards for reliability; no sacrifices on validity or quality.
  \end{itemize}
  \item Support \& training materials used first by our team members; external capacity development externally where funded through projects \& NMOs.
\end{itemize}

\medskip
Other research software has large teams of dedicated developers:
\begin{itemize}
  \item MATLAB — developer MathWorks has 5000 employees.
  \item Tableau — 4000 employees.
  \item pandas Python package — \href{https://pandas.pydata.org/docs/whatsnew/v1.4.0.html\#contributors}{100s of contributors}.
\end{itemize}

\end{frame}


\begin{frame}
\frametitle{Why work towards openness?}

Scientific knowledge and the means to produce it should be accessible.

\medskip
Climate change \& sustainability problems are urgent.
\begin{itemize}
  \item Barriers \& inclusion: make it easier, not harder, for a wider circle of experts to use high-quality tools to analyse energy systems and provide advice to their own communities.
  \item Efficiency: avoid waste in re-implementing the same methods many times.%
  \footnote{cf. the example of \href{https://en.wikipedia.org/wiki/Linux_kernel\#Estimated_cost_to_redevelop}{the Linux kernel}.}
\end{itemize}

\medskip
Neighbouring research disciplines have already gone this route.
\begin{itemize}
  \item e.g. climate \& Earth system modeling; economics.
  \item Unlocked more advanced \& useful research, better validity.
\end{itemize}

\end{frame}


\begin{frame}
\frametitle{Key practices, in brief}

\begin{description}
  \item [Modularity:] group code into packages (modules, classes, functions) each with a simple, clear purpose (data storage, model framework, etc.) → reusable, interoperable with each other \& with ‘upstream’ software.
  \item [Docs:] code comes with plain text explanation of how to use it.
    Further examples, tutorials, etc. for beginners.
  \item [Testing:] code is continuously tested to ensure expected, valid outputs.
    Frequent adjustment to keep up with evolving “upstream” software; on multiple operating systems.
  \item [Releases:] regular; every 6 months or less.
    e.g. \texttt{ixmp} and \texttt{message\_ix} 3.5.0 on May 6.
    Changes clearly described to help users migrate.
\end{description}
\note{
Another way to think of this is that MathWorks' has, probably, an order of magnitude more people 100\% focused on support than the MESSAGE\emph{ix} has doing \emph{all} development.
}

\end{frame}

\begin{frame}
\frametitle{The MESSAGE\emph{ix} stack}

\begin{columns}[T]
\column{49mm}
\begin{tikzpicture}[
  every node/.style = {
    ultra thick,
    draw = IIASAblue,
    anchor = south,
    rounded corners = 2 mm,
    minimum width = 47 mm,
    outer sep = 2 mm,
  },
]

  \onslide<2-4>{
    \node (ixmp) {\ttfamily ixmp};
  }
  \onslide<1>{
    \node [ fill = IIASAteal ] (ixmp) {\ttfamily ixmp};
  }

  \onslide<1,3,4>{
    \node (messageix) at (ixmp.north) {\ttfamily message\_ix};
  }
  \onslide<2>{
    \node [ fill = IIASAteal ] (messageix) at (ixmp.north) {\ttfamily message\_ix};
  }
  \onslide<1,2,4>{
    \node (message-ix-models) at (messageix.north) {\ttfamily message-ix-models};
  }
  \onslide<3>{
    \node [ fill = IIASAteal ] (message-ix-models) at (messageix.north) {\ttfamily message-ix-models};
  }
  \onslide<1-3>{
    \node at (message-ix-models.north) {\ttfamily User code};
  }
  \onslide<4>{
    \node [ fill = IIASAteal ] at (message-ix-models.north) {\ttfamily User code};
  }

\end{tikzpicture}


\column{100mm}
\only<1>{
\structure{\ttfamily ixmp}: data storage and solver interface

\smallskip
Docs: \href{https://docs.messageix.org/ixmp}{docs.messageix.org/ixmp} \\
Code: \href{https://github.com/iiasa/ixmp}{github.com/iiasa/ixmp}

\smallskip
\begin{itemize}
  \item e.g. $Y = f(m, b; x)$
  \item ixmp provides data handling for $N$-dimensional variables ($x$), parameters ($m, b$), models (any equations, e.g. $f(\ldots)$) implemented in e.g. GAMS; and invoking these to produce/store model outputs ($Y$).
  \item Tools and utility code for \emph{general} modeling, e.g. post-processing.
\end{itemize}
}
\only<2>{
\structure{\ttfamily message\_ix}: the MESSAGE implementation

\smallskip
\href{https://docs.messageix.org/ixmp}{docs.messageix.org} \structure{\bfseries —always start here.}
\href{https://github.com/iiasa/message_ix}{github.com/iiasa/message\_ix}

\smallskip
\begin{itemize}
  \item $Y = a_{d_1,d_2} x_1 + b_{d_2,d_3} x_2 + c_{d_1,d_2,d_3}$
  \item Specific equations (MESSAGE) in GAMS, but no fixed/provided values for parameters → a \emph{framework} for models, not a specific model.
  \item Tools for \emph{energy system/IA modeling} related to the MESSAGE format, e.g. for sets of ‘vintage’ and ‘active’ periods for technology construction/use; to prepare data for $a$ or $b$, etc.
\end{itemize}
}
\only<3>{
\structure{\ttfamily message-ix-models}: tools/data for the MESSAGE\emph{ix}-GLOBIOM family

\smallskip
\href{https://docs.messageix.org/models}{docs.messageix.org/models}
\href{https://github.com/iiasa/message-ix-models}{github.com/iiasa/message-ix-models}

\smallskip
\begin{itemize}
  \item A \emph{family} of related global/single-country models.
  \item Specific lists of e.g. regions, technologies, commodities that give these models a (mostly) common structure.
  \item Handling code for some ‘upstream’/input data sources.
  \item Tools for preparing specific model representations used by IIASA ECE.
\end{itemize}
}
\only<4>{
\structure{‘User’ code}: specific models, projects, \& research

\smallskip
\begin{itemize}
  \item May be public, private (permanent, or temporarily).
  \item e.g. models with new details; specific scenarios; model ‘variants’ like MESSAGE\emph{ix}-Buildings, -Materials, and -Transport.
  \item Code can be contributed to \texttt{message-ix-models} at any point.
\end{itemize}

\smallskip
\structure{\ttfamily message\_data}: IIASA ECE's internal/private repository for in-progress MESSAGE\emph{ix}-GLOBIOM work.
}

\end{columns}

\end{frame}

\begin{frame}
\frametitle{Welcome—please dive in!}

\structure{\bfseries Use} the software to do great research.

\medskip
As you do, we value feedback, tips, and input:

\begin{itemize}
  \item Join the discussion forum at \url{https://github.com/iiasa/message_ix/discussions};\\ exchange mutual support with other users.
  \item Contribute your code, bug reports; help shape new features.
  \item Give us feedback on the “Pre-requisite knowledge \& skills” list in the documentation: does this clearly advertise the learning curve for energy systems modeling?
\end{itemize}

\end{frame}

\section{Feedback}

\begin{frame}
\frametitle{MESSAGE\emph{ix} user experience (UX)}

Thanks to 16 participants who filled out the survey! More welcome.

\bigskip
Recap: limited \structure{resources (money \& staff time)} to divide across \emph{use} (IIASA ECE core research), \emph{development}, and \emph{support} activities.

\bigskip
To increase resources for support and development—multiple options:
\begin{itemize}
  \item IIASA staff: \structure{dedicated funding}, e.g. parts of project budgets.
  \item \structure{In-kind}: staff hosted/paid/contracted by community orgs (not IIASA).
  \item \structure{Ad-hoc}: Payment-for-service, support subscriptions, etc.
\end{itemize}
\hspace{20mm}…any others?

\end{frame}

\begin{frame}[plain]
\frametitle{Your degree of experience with…}
\includegraphics[height=0.8\textheight]{2022-05-10-messageix-users.png}
\end{frame}

\begin{frame}[allowframebreaks]
\frametitle{Specific responses}

Questions we should ask about any wished-for enhancement, idea, or request:

\bigskip
Are they…
\begin{itemize}
  \item Incremental \& light-weight → easier, quick, most likely to implement.
  \item Large time investment (days, weeks, months of staff time) → less likely / unlikely without a concrete resource commitment.
\end{itemize}

\bigskip
Is the \texttt{message\_ix} software/docs the right place for these?

\bigskip
By whom and how will the resources be supplied?

\framebreak

\structure{Documentation}

\medskip
Suitable for needs?
6 — yes; 7 — a little, 0 — not at all

\medskip
Suggested improvements?

\begin{quote}
  [The] TIMES Model [has] nicely \structure{explained each concept of energy modelling} and then provided the associated parameters. At many places the documentation on MESSAGEix does not explain the necessary parameter. It was really difficult to actually understand the underlying concept behind a parameter.
\end{quote}

\begin{quote}
  Adding \structure{examples to each section} for specific indication towards use of a parameter.
\end{quote}

\framebreak

\structure{What was your experience when you first started using MESSAGE\emph{ix}?}

\begin{quote}
  To build a real world energy system takes more extra time since both \structure{model structure and input data needs substantial work and time}.
\end{quote}

\begin{quote}
  Comprehensive but complex.
\end{quote}

\begin{quote}
  Knowledge of Python/GAMS/R/Conda restricts potential users. Need a better interface so that \structure{even a person with less programming knowledge} can use and understand the model.
\end{quote}

2 comments re: installation.

\framebreak

\structure{Contributing to development}

\medskip
Would you? 9 — yes; 3 — maybe; 4 — no.

\medskip
Features/support options of interest:

\begin{quote}
  Unit commitment simulation — Power procurement — Power sector-related features are missing.
\end{quote}

\begin{quote}
  Video tutorials \& more technical docs — Video classes — Beginners' YouTube channel.
\end{quote}

\begin{quote}
  Develop a tutorial where \structure{all} the parameters are utilized.
\end{quote}

\end{frame}

\end{document}
