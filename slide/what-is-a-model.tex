\begin{frame}[allowframebreaks]{What is a model?}
\framesubtitle{Three perspectives and resulting insights}

A \structure{knowledge object} that embodies or represents a theory or understanding of some real-world phenomenon.

\medskip
\begin{itemize}
  \item Theories often causal.
  \item Relationships expressed quantatively: equations connecting variables representing concepts measured in certain, systematic ways.
  \item In large-scale integrated assessment, systematized concepts often aggregate: GDP, country, sector.
\end{itemize}

\framebreak
A \structure{scientific instrument}\footcite{omalley-2019} that is used to perform experiments: “What would be the outcome (effect on quantity $Y$) if $X$ were changed from $x_1$ to $x_2$?”

\medskip
\begin{itemize}
  \item Another instrument: the Large Hadron Collider (LHC).
    \begin{itemize}
      \item EUR 7.5 billion budget; labour from many specialized roles.
      \item Components for preparing the experiment, running it, measuring outcomes are carefully designed, constructed, tested.
    \end{itemize}
  \item Instruments require meticulous attention to detail.
  \item Description of methods includes description of instruments, so the experiment can be reproduced.
\end{itemize}

\framebreak
A \structure{software project} in which people in organizations create code that is run on computer systems.

\medskip
\begin{itemize}
  \item All software has bugs; all organizations have politics.
  \item Software is constantly evolving and never complete.
  \item Tendency to overinvest time in new code vis-à-vis quality \& docs.
  \item “Technical debt”: code grows stale over time.
\end{itemize}

\med
\structure{\bfseries But!} good software development practices exist, and are widely used to ensure that software meets needs.

\end{frame}
