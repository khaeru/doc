\documentclass[12pt,aspectratio=169]{beamer}
\usetheme[version=2024]{iiasa}

\usepackage[
  maxnames = 1,
  style = authoryear,
  giveninits,
  terseinits,
  maxcitenames = 3,
  ]{biblatex}
\addbibresource{all.bib}

\setcounter{secnumdepth}{3}

% Disable section title pages from 'iiasa' theme
\AtBeginSection{}

\title{(Transport × Demand) in IPCC AR7}
\subtitle{Panel discussion response}
\institute{
  WG 1 Transport / \\
  Energy Demand changes Induced by Technological and Social innovations (EDITS) network}

\date{
  \texorpdfstring{EDITS 2025-Q1 Meeting — Mon 24–Tue 25 March 2025}%
  {2024-10-10}}

\author{\texorpdfstring{Paul Kishimoto, Luis Martinez, John Pritchard\\
  \href{mailto:kishimot@iiasa.ac.at}{\ttfamily \scriptsize <kishimot@iiasa.ac.at>}%
  }{Paul Natsuo Kishimoto <kishimot@iiasa.ac.at>}}

\begin{document}

\maketitle

\begin{frame}
\frametitle{(Transport × Demand) in IPCC AR7}

\structure{Responding} to the two panels:
\begin{itemize}
  \item “What are the EDITS-relevant content expectations of AR7?”
  \item “What can and should EDITS contribute to IPCC AR7?”
\end{itemize}

Framed as \structure{obstacles} (to a high-quality assessment) and \structure{actions}:

\tableofcontents

\end{frame}

\section{1. Freight activity is underrepresented}
\begin{frame}
\frametitle{1. Freight activity is underrepresented}
As often reminded, freight activity accounts for $~$half of transport energy use and GHG emissions (‘E\&E’).

Accordingly, options for its abatement should get roughly equal attention to passenger activity—but don't.

Some mode-specific work arranged around policymaking activities in ICAO (aviation) and IMO (shipping)—but connections to EDITS-type activities are tenuous.

\bigskip
Suggested \structure{actions}:
\begin{enumerate}
  \item [1A] Promote and collect work on freight E\&E impacts that are consequent of changes in demand, consumption, etc.
  \item [1B] Make a \structure{themed/focused call for EDITS FT} activities in 2026 and beyond—i.e. top-down rather than bottom-up selection.
\end{enumerate}

\end{frame}

\section{2. Global-scope work is mostly “grey literature”}
\begin{frame}[allowframebreaks]
\frametitle{2. Global-scope work mostly “grey literature”}
Organizations like (ITF-OECD) doing global-scope transport E\&E assessment usually \structure{do not} face incentives to put resources into peer-reviewed publication.
\begin{itemize}
  \item \emph{Some} of this occurs, but treated as a side activity.
  \item → fewer sources; IPCC authors must argue for “grey literature” exemptions.
\end{itemize}

\medskip
In contrast: small-/medium-size academic teams face incentives for ‘novelty’ of methods, but not on careful \& exhaustive data collection, comparison, alignment with existing studies.

\framebreak
Suggested \structure{actions}:
\begin{enumerate}
  \item [2A] Coordinate \structure{invited/special issues} in receptive journals to facilitate peer-reviewed references for work that risks being consigned to “grey literature”:
    \begin{itemize}
      \item Allow for ‘lightweight’/model description-type papers with referenced data and documentation → more compatible with organizations already publishing projections.
      \item Especially targeted to themes where we are aware of literature gaps, e.g. freight per (1A) above.
    \end{itemize}
  \item [2B] Organize 1 or more \structure{workshop(s) or symposia}.
    \begin{itemize}
      \item Possibly attached to an \structure{iTEM} (International Transport Energy Modeling, \href{https://transportenergy.org}{transportenergy.org}) annual meeting or other event.
      \item Participation could explicitly align with publications (2A above) to follow.
    \end{itemize}
\end{enumerate}

\end{frame}

\section{3. G-/NTEM alignment is poor due to historical data quality}
\begin{frame}[allowframebreaks]
\frametitle{3. Poor alignment due to historical data quality}
Persistent and perennial issue:
\begin{itemize}
  \item Flagged by \textcite{yeh-2016} and earlier.
  \item A problem in IPCC AR6 WGIII ‘Transport’ chapter.
  \item Visible in ScenarioMIP/CMIP7/SSP update activities.
\end{itemize}
…pointing to lack of resources and perceived importance.

\bigskip
Suggested \structure{actions}:
\begin{enumerate}
  \item [3A] Public, rolling (frequent iteration) process on key measures for data up to the present.
    \begin{itemize}
      \item Facilitates debate over \structure{common} or \structure{acceptable alternate} choices on handling data gaps and discrepancies.
      \item Methods and values for accounting for “COVID dip” etc.
      \item Building on experience in the EDITS MCE.
    \end{itemize}
\end{enumerate}

\framebreak
\begin{enumerate}
  \item [3B] Establish common methods/standards for alignment:
    \begin{itemize}
      \item Start with the crude base-year indexing used in AR6.
      \item Iterate and improve.
      \item Allow would-be contributors to use these shared tools to check their own projections.
      \item Also publicly specify and call for ‘diagnostic’ used to check or facilitate alignment.
    \end{itemize}
\end{enumerate}

\end{frame}

\section{4. Harmonization of regional-/national-scope work is underresourced}
\begin{frame}
\frametitle{4. Regional-/national-scope harmonization}

There are many national- and multi-country-scope models of transport activity, with some or richer links to E\&E and other impacts (esp. air pollution \& health).

\smallskip
Although often \structure{more represent\textbf{ative}}, this work was \structure{poorly represent\textbf{ed}} in AR6:
\begin{itemize}
  \item Scenario \& policy assumptions, input data that are appropriate in-scope (e.g. national official GDP projections) but hard to compare.
  \item Page, word, and figure count limits limit the space devoted to evidence on any one region or country.
  \item Not enough existing work (peer-reviewed or otherwise) that overcomes these issues to provide synthesis.
\end{itemize}

\bigskip
Suggested \structure{actions}: ??? unclear —perhaps same mechanisms as 2B, 3A, 3B?
\end{frame}

\section{5. AR6 WGIII data mechanisms excluded transport demand data}
\begin{frame}[allowframebreaks]
\frametitle{5. AR7 mechanisms for transport demand data}

In AR6, calls for data, systems, formats, and processing were mainly established by WGIII Chapter 3 / IAM community.

\smallskip
Much data from transport demand-focused research did not make it \structure{into or through} these pipelines where:
\begin{itemize}
  \item Researchers were not aware of these calls and their deadlines.
  \item Data were for measures, spatial scopes, or resolutions not imagined by the “IAMC template.”
  \item Researchers were not familiar with the data submission process or formats, and lacked tools to prepare data.
  \item Data did not align with Ch.3 processing methods, e.g. climate outcome scoring.
\end{itemize}

\framebreak
Suggested \structure{actions}:
\begin{enumerate}
  \item [5A] Flexible, open mechanisms for researchers to flag their data as AR-relevant.
    \begin{itemize}
      \item IAM-related requirements only for data from IAMs, or for data destined for certain pipelines.
    \end{itemize}
  \item [5B] Dedicated resources to support transport data providers (esp. on possible demand-side transformations) in overcoming barriers:
    \begin{itemize}
      \item Outreach; assistance in publishing data in interoperable, reusable formats \& with adequate metadata.
      \item “Trial runs” of assessment processes with clear feedback on where data “drops out” (is excluded).
      \item Facilitated discussion on measures, data structures, tools, etc.s
    \end{itemize}
\end{enumerate}

\end{frame}

\makefinalslide

\appendix

\begin{frame}[allowframebreaks]
\frametitle{References}

\printbibliography[heading=none]

\end{frame}

\end{document}
