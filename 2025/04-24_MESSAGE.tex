\documentclass[12pt,aspectratio=169]{beamer}
\usetheme[version=2024]{iiasa}

\usepackage[
  maxnames = 1,
  style = authoryear,
  giveninits,
  terseinits,
  maxcitenames = 3,
  ]{biblatex}
\addbibresource{all.bib}

\setcounter{secnumdepth}{3}

% Disable section title pages from 'iiasa' theme
\AtBeginSection{}

\title{MESSAGE development/research and ‘AI’ coding tools}

\date{
  \texorpdfstring{2025-04-24 MESSAGE team meeting}%
  {2025-04-24}}

\author{\texorpdfstring{Paul Natsuo Kishimoto\\
  \href{mailto:kishimot@iiasa.ac.at}{\ttfamily \scriptsize <kishimot@iiasa.ac.at>}%
  }{Paul Natsuo Kishimoto <kishimot@iiasa.ac.at>}}

\begin{document}

\maketitle

\begin{frame}
\frametitle{These slides}

\bigskip
According to some prognosticators, “Artificial General Intelligence” will take over all research in 2–5 years;
we will all be unemployed;
AGI will solve climate change on its own;
and there's nothing we can do to stop this.

\smallskip
According to others, “AI slop” will pollute the Internet,
most code,
research products,
and data to the extent that these are largely useless.
Most people will lose the capability to independently discern truth,
and so research performed by the remaining few will be even more valuable than today.

\bigskip
This deck is not about either of those scenarios.

\medskip
It's focused on our day-to-day work, and tries to frame a pragmatic discussion about whether and how we want to (or should) use ‘AI’ coding tools in our work.

\end{frame}

\begin{frame}
\frametitle{What are we trying to do with MESSAGE \& co.?}

Research that is of \structure{high quality} and is \structure{scientific}:
\begin{itemize}
  \item Valid.
  \item Defensible and explainable.
  \item Useful and impactful.
  \item Often ‘novel' in some sense.
\end{itemize}

\bigskip
It may not always \emph{feel} this way:
under the pressure of deadlines,
we may feel that we trying to
\structure{‘just’ produce some output—anything}.

But this guiding purpose should frame our decisions.

\end{frame}

\begin{frame}
\frametitle{Validity and quality}
In order to achieve these,
we need to ensure the following are all consistent:
\begin{enumerate}
  \item Actual phenomena in the real world.
  \item Our abstract understanding of (1).
  \item Mathematical or symbolic models of (2).
  \item Specific quantifications of (3).
  \item Code to implements (4).
  \item Descriptions of the code (5)
    —tests, docs, oral remarks, etc.
\end{enumerate}

\medskip
At every link in this chain there is the possibility for error.

These errors are what undermine validity/quality.

Errors are more likely if we seek ‘novelty’ at any stage.
\end{frame}
\note{
Often,
because we aim for 'novelty',
(2) and (3) are entirely new.
There's always the possibility to make errors when doing something new,
and we all have the experience of thinking carefully and slowly about something
before putting it to, for instance, peer review or a thesis defense.

Other times novelty is not in (2–3) but in (4–5).
The same applies.
}

\begin{frame}
\frametitle{Our context}

\begin{itemize}
  \item Team of 20+ with \structure{turnover}
    (people join and leave).
  \item Many \structure{simultaneous projects} ongoing.
  \item Usually each project involves writing new code
    and/or using “the model” (in fact, many scenarios with critical differences) in new ways.
  \item Reuse of/iteration on ‘capabilities' from older projects.
  \item Varied amount of technical training of each colleague.

    (More on this later.)
  \item Loose/informal time tracking, if any.

    Time spent by 2+ team members on shared tasks is not accounted or analyzed.
\end{itemize}
\end{frame}

\begin{frame}
\frametitle{Our day-to-day work}

One way to frame our individual and collective activities
is as time spent answering various questions:

\begin{enumerate}
  \item What parameter(s) should we use to implement Scenario X?
  \item What data is available?
    How can/should we transform it to meet our needs?
  \item Do we already have code that implements Method Y?
  \item Where is it? How does it work?
  \item Will it work for \emph{my} application?
  \item The results seem unexpected.

    How do I understand whether and why they are correct?
  \item What was the original thought-process/intent behind this code?
  \item \structure{Who} wrote this code/prepared this data?
\end{enumerate}
\end{frame}

\begin{frame}
\frametitle{Questions → communication → answers}

Often we answer the \emph{last} question (“Who?”) \emph{first},
and then go talk to the person(s) responsible.

\bigskip
It's nice to talk to people, but as a primary approach this has risks and costs:

\begin{itemize}
  \item The person may no longer be around.
  \item They may be fully preoccupied with other tasks,
    unable to make time for us (a) now or soon,
    or (b) sufficient to answer all our questions.
  \item They may not know our project context.
  \item It may be a long time since they worked on the code/data,
    so answers are no longer fresh in mind.
\end{itemize}
\end{frame}

\begin{frame}
\frametitle{Asynchronous/passive modes of communication}

These are very often%
\footnote{Of course, not always}%
more \structure{efficient} and \structure{effective}.

\bigskip
They include modes like:
\begin{itemize}
  \item Threads on Slack/Teams.
  \item Documentation.
  \item Things in the code itself: comments, tests.
  \item Git history and ‘blame’.
  \item GitHub issue and PR discussions.
  \item Presentation materials: papers, slides, tutorials.
\end{itemize}
\end{frame}

\begin{frame}
\frametitle{How to think about ‘efficiency’}

Compare \structure{Scenario A}:
\begin{itemize}
  \item Information is in the head of Person A.
  \item Persons B, C, D, and E each need that info
    —at different times separated by weeks or months.
  \item Person A must take 30+ minutes to explain each time.
  \item 4 convos × 2 people × 0.5 h = 4 person-hours spent.
\end{itemize}

\medskip
…with \structure{Scenario B}:
\begin{itemize}
  \item Person A spends 60 minutes writing a careful, clear, complete description.
  \item Persons B, C, D, and E each spend 20 minutes reading it.
  \item 1 + 4 × (1/3) = 2.3 person-hours spent,

    including half as much total time from Person A.
\end{itemize}
\end{frame}

\begin{frame}
\frametitle{More context: we operate at or near capacity}

Very often we run into situations where:
\begin{itemize}
  \item The person who created code is not available
    (because departed or too busy) to answer Qs about it.
  \item We aren't even aware that code has already been created for task X.

    We end up spending time on a 2nd or 3rd implementation.
  \item We can't tell if existing code can be applied to a new use.
  \item Scrutiny (eg. peer review) prompts us to carefully inspect,
    reconsider and/or tweak the methods in some code/data.
\end{itemize}

\medskip
We have few spare resources (time)
with which to overcome these hurdles.

\medskip
When those run out, we are forced into workarounds,
‘hacks',
reducing scope,
pushing back timelines,
overtime,
and other \structure{undesirable responses}.
\end{frame}

\begin{frame}
\frametitle{Contribution standards}

The reason we have these is to \structure{\bfseries make our work more intelligible} to ourselves.

\medskip
The standards are designed to:
\begin{itemize}
  \item make it easy for any team member to look at their own or another's work,
  tomorrow or in 5 years;
  \item help that person understand all the 6 steps \structure{well enough to make decisions} like
    “Can I use this?" or “(How) Should I revise this?";
  \item aid with \structure{self-discipline} and development of good habits and skills;
  \item result in \structure{consistent forms} of information: ones that new colleagues can learn—%
    easily and only once—%
    to read.
\end{itemize}

More broadly, they can help remind us that our work is not merely to \structure{“crank something out”} today, but also to prepare a \structure{usable and useful toolkit} for our own and teammates' future work.
\end{frame}

\begin{frame}
\frametitle{More context: skills and training}

Topics like programming and software engineering are \structure{very unevenly included} in undergrad/graduate curricula.
Through no fault of their own,
people join our team having learned a little, or a lot, about these.

\bigskip
Result: following the standards is initially \structure{easier for some}, harder for others
—even if they want to and are quick to learn.

\bigskip
We \structure{balance costs and benefits} when we decide on contribution standards:
\begin{itemize}
  \item Some colleagues will need to spend time to help others learn our shared practices and concepts%
    —both current practices and any new ones.
  \item Investing in quality now brings greater benefits/saving in the long term…
  \item …but we don't have $\infty$ time to help all meet arbitrarily high standards.
\end{itemize}
\end{frame}

\begin{frame}
\frametitle{What does any of this have to do with ‘AI’?}

Some things that are \structure{always true}:
\begin{itemize}
  \item As members of a team, we have responsibilities and obligations to our team members.
  \item Actions have consequences, for oneself and for others.
  \item Some choices make it more costly to understand, apply, maintain, and develop our research code/data.
  \item It is wise to consider a choice of tools from the standpoint of
    whether they offer the affordances\footnote{features, roughly} to do the work required.
    \begin{itemize}
      \item Example: asked to carve a wooden doorknob,
        you reach for a chainsaw.

        This may ‘work’, but probably adds needless difficulty.
    \end{itemize}
\end{itemize}

\medskip
These \structure{remain true} whether or not ‘AI’ coding tools are available,
or whether one chooses to use them.
\end{frame}

\begin{frame}
\frametitle{Observed failure modes of ‘AI’ coding tools}

Only a few we've already observed:

\begin{enumerate}
  \item The tools will \structure{‘confidently’} produce incorrect responses.
  \item Reliance on the tools inhibits users' ability for independent cognition.%
    \footnote{\href{https://doi.org/10.1145/3706598.3713778}{doi: 10.1145/3706598.3713778}}
\end{enumerate}

\bigskip
For example: Suppose you prompt a tool, “Write a Python function that implements the method described in \structure{Smith \& Jones (2022)}.”

\medskip
The tool may be able to locate and input the paper, and output code that doesn't error.

\medskip
However: since you have not read the paper
and understood the method yourself,
you may be unable to inspect the code
and realize there are subtle mistakes in the implementation.
\end{frame}
\note{
There's not enough time to recap all of these,
much less get into the ethical, moral,
and social issues surrounding them.
}

\begin{frame}
\frametitle{From individual → team impacts}

Next, suppose you put forward the tool's output as your own work.

It appears in a PR that must be reviewed, used, and maintained by colleagues.

\medskip
During this process, you may be unable to:
\begin{itemize}
  \item provide useful answers to questions about \emph{how} you chose to implement the method.
  \item communicate (in docs, tests, even orally) about \structure{limitations} and applicability of the code and method.
\end{itemize}

\medskip
The impact is \structure{extra work}—potentially lots of it—for your colleagues.

\medskip
You might, for instance, force a colleague who reviews the code to \emph{themselves} read Smith \& Jones (2022);
understand the method;
read through the ‘AI’ tool's output;
and identify where the code fails to implement it.
\end{frame}
\note{
This is an attempt to shift your own work into someone else.
Sometimes reassigning tasks within a team is a smart choice!
But that should be discussed directly and agreed together.
}

\begin{frame}
\frametitle{More specific failure modes}

\begin{enumerate}
  \item [3.] The tools don't appear able to digest and conform to our code style,
    Git/docs conventions,
    and PR templates.
  \item [4.] The code usually makes poor use of existing utilities,
    leading to bloat and duplication
    that could make maintenance impossibly costly.
\end{enumerate}

\medskip
For (4), suppose we have utility functions that do tasks A, B, and C.

\medskip
A tool prompted to write code for D will write a very long function that internally duplicates A, B, and C, and does not call them or even comment “These lines are the same as A/B/C.”

\medskip
Allowing this (and future duplicates for E, F, G, etc.) into our code base forces us to later waste time puzzling over differences between these.
It can add 100s or 1000s of needless lines of code, and is not parsimonious.
\end{frame}

\begin{frame}
\frametitle{More uncertain failure modes}

Because these are uncertain,%
\footnote{in these sense of \emph{how}, not \emph{if}.}
they create further costs for us to investigate,
identify concrete risks/impacts,
and develop/promote ways (if possible) to avoid those.

\medskip
\begin{itemize}
  \item Legal, e.g. licensing and copyright of training material/output.
  \item Moral, ethical, and social.
  \item Environmental.
  \item Financial, i.e. costs and fees.
\end{itemize}
\end{frame}

\begin{frame}
\frametitle{What can we do? Some suggestions}

\begin{enumerate}
  \item \structure{Remember}:
    \begin{itemize}
      \item “I am part of a team.”
      \item “My choices have impacts on my teammates.”
      \item “I am ultimately responsible for my own work,
      no matter which tools I use to produce it.”
    \end{itemize}
  \item \structure{Consider/factor-in} the clean-up time needed to refine contributions of acceptable quality from the raw output of ‘AI’ tools.
    \begin{itemize}
      \item This involves learning (yourself) what “acceptable quality” means,

        and knowing how to discern when ‘AI’ does not produce it.
    \end{itemize}
  \item \structure{Be purposeful and strategic} about your own skills development.
    \begin{itemize}
      \item Skills to read/understand/implement research methods are valuable.
      \item Time and energy spent learning how to prompt a tool to mimic these things is time and energy you can not spend improving your own skills.
    \end{itemize}
\end{enumerate}
\end{frame}

\begin{frame}
\frametitle{What can we do? More + your suggestions}

\begin{enumerate}
  \item [4.] \structure{Disclose up front} any text or code produced using an ‘AI’ tool:
  indicate which tool(s) you used, and how.
    \begin{itemize}
      \item This alerts colleagues that there may be extra cost in reviewing your work.
    \end{itemize}
  \item [5.] \structure{“Request changes”} on PRs with ‘AI’ content that fail to follow standards.
    \begin{itemize}
      \item This feels rude, and we don't like to do it.

        But maintainers do not have time to do the extra work that is created.
    \end{itemize}
  \item [6.] \structure{Update our Code of Conduct} to include some of the above.
  \item [NOT:] Entirely prohibit use of ‘AI’ tools.
    We can't afford the cost to police this.
\end{enumerate}

\bigskip
\structure{\bfseries What do you think?}

\medskip
How can we \structure{as a team} guard against unsustainable costs if/as colleagues choose to experiment with these tools?
\end{frame}

\makefinalslide

\end{document}
