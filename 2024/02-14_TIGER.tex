\documentclass[12pt,aspectratio=169]{beamer}

\usetheme[version=2019]{iiasa}

\usepackage{xcolor}
\usepackage{minted}
\setminted{
  fontsize=\footnotesize,
}
\renewcommand{\mod}[1]{\mintinline{python}{#1}}
\newcommand{\func}[1]{\mintinline{python}{#1()}}
\newcommand{\py}[1]{\mintinline{python}{#1}}

\usepackage{emoji}

\title{Test-wrItinG for rEproducible Research \emoji{tiger}}
\institute{IIASA Energy, Climate, and Environment program}

\date{Wednesday, 14 February 2024}

\author{\texorpdfstring{Paul Natsuo Kishimoto \scriptsize\newline
  \href{mailto:kishimot@iiasa.ac.at}%
       {\ttfamily <kishimot@iiasa.ac.at>}}%
  {Paul Natsuo Kishimoto <kishimot@iiasa.ac.at>}}

\begin{document}

\maketitle

\section{Introduction}
\begin{frame}{Outline}
  \tableofcontents
\end{frame}

\begin{frame}[allowframebreaks]{Motivation, goals, expectations}
Motivation:

\bigskip
We would like to do \structure{\bfseries research that provides a valid, scientific basis} for other research or policy-making.
\begin{itemize}
  \item Models and analysis implemented in reproducible, reusable, extensible code.
  \begin{itemize}
    \item Code that behaves in well-defined ways, as documented and expected.
    \begin{itemize}
       \item Testing, continuous integration (CI), and other quality control (QC) practices.
    \end{itemize}
  \end{itemize}
\end{itemize}

\pause
Reduce time costs of debugging pieces of code (→models→research) within overall workflows (even if those are not reproducible), including:
\begin{itemize}
    \item
\end{itemize}

\framebreak
Expectations: Participants should…
\begin{itemize}
  \item know Python, pandas, the MESSAGEix stack, and basic usage of a command-line interface.
  \item have desire and intent to learn.
\end{itemize}

\medskip
Objectives: Participants will…
\begin{itemize}
  \item feel comfortable using pytest in daily development.
  \item read and comprehend the MESSAGEix-stack test suites, \mod{.testing} modules, and documentation.
  \item use extant tests as a guide to emulate in writing new tests.
  \item communicate clearly and fluently about testing while developing new research code.
\end{itemize}
\end{frame}

\section{Test-driven development with pytest: basics}
% How to use pytest efficiently when developing: beginner/intermediate/advanced
\begin{frame}[allowframebreaks,fragile]{Example: writing a new function}

We begin with a textual \structure{description of what we want the code to do}.

\medskip
When writing new code, we \emph{always} have such a description, although sometimes it is only implicit (if so: speak it to your \emoji{baby-chick}).

\medskip
For example: Let's write a function that receives a pd.Series and returns a pd.DataFrame with both the original values, and the originals divided by two.

\framebreak
We start by writing \emph{only} the \structure{function signature}:
\begin{minted}{python}
import pandas as pd

def orig_and_half(input: pd.Series) -> pd.DataFrame:
    """Return `input` and half of `input` as two columns."""
    raise NotImplementedError
\end{minted}
Already at this stage, \structure{type hints} can be exactly that: a hint to ourselves.
If our IDE supports using a type checker like `mypy', we will get real-time feedback as we type out this function if we:
\begin{enumerate}
  \item Do things to \verb|input| that are not valid on a pd.Series.
  \item Fail to return the expected pd.DataFrame.
\end{enumerate}

Even if we forgot to write the \structure{docstring}, the hints remind what the function should do.
\end{frame}

\begin{frame}[fragile]{Write a minimal test \textbf{first}}
We \structure{immediately} create a matching file/submodule for tests.

\medskip
If our original code was in \verb|message_ix_models.tools.foo|, we create our test module at \verb|message_ix_models.tests.tools.test_foo|. This parallel name makes the tests easy to find from the code and vice versa.

\medskip
The parallelism continues in the name of the test function:
\begin{minted}{python}
from message_ix_models.tools.foo import orig_and_half

def test_orig_and_half():
    orig_and_half()
\end{minted}

\medskip
Quiz: what will happen if we run this test?\pause

→ \structure{TypeError}, because we call the function with no arguments when 1 is required.
\end{frame}

\begin{frame}[fragile]{Run the test}
For this, we can use:
\begin{itemize}
    \item \href{watchexec}{https://watchexec.github.io}, a handy tool that re-runs a command on any change to certain files.
    \item The \href{https://docs.pytest.org/en/stable/reference/reference.html\#command-line-flags}{\texttt{pytest -k} command-line option}.
\end{itemize}

\begin{verbatim}
watchexec -e py -- "pytest -k 'orig_and_half' -rA -vv"
\end{verbatim}
This command:
\begin{itemize}
  \item Re-runs automatically on changes to files with the "py" extension.
  \item Runs only tests that have \verb|orig_and_half| in the name.
  \item Reports "A"ll output—even for passing tests—and is "v"ery "v"erbose.
\end{itemize}
\end{frame}

\begin{frame}[allowframebreaks,fragile]{Complete the test}
We can do this incrementally.
Each time we save the file with changes, \verb|watchexec| and \verb|pytest| show us how we're doing.

\medskip
First, give the function some input:
\begin{minted}{python}
import pandas as pd

def test_orig_and_half():
    # Create input data
    input = pd.Series([1, 2, 4])

    # Function runs
    orig_and_half(input)
\end{minted}

\framebreak
Next, store and check its output:
\begin{minted}{python}
import pandas as pd

def test_orig_and_half():
    # Create input data
    input = pd.Series([1, 2, 4])
    # Expected corresponding output
    exp = pd.DataFrame([[1, 0.5], [2, 1.0], [4, 2.0]])

    # Function runs
    result = orig_and_half(input)

    # Result is as expected
    assert exp == result
\end{minted}

\framebreak
And improve by using \structure{test utilities}, in this case from \href{https://pandas.pydata.org/pandas-docs/stable/reference/api/pandas.testing.assert_frame_equal.html}{pandas.testing}.

\begin{minted}{python}
...
import pandas.testing as pdt

def test_orig_and_half():
    ...

    # Function runs
    result = orig_and_half(input)

    # Result is as expected
    pdt.assert_frame_equal(exp, result)
\end{minted}
\end{frame}

\begin{frame}{Write code!}
Now we can dive in on actually writing our function \mintinline{python}{orig_and_half()}.

\medskip
At some point, we will notice that our test is passing, and find that we have completed the function.

\smallskip
Exercise: Do this yourself.

\medskip
As we expand the function's features (what we expect it to do), we can:
\begin{itemize}
  \item Add or change the inputs and expected outputs in the test function.
  \item Adjust the function signature and the way it is called in the test.
  \item Parametrize the test function or add additional test functions to test different combinations of arguments.
\end{itemize}
\pause\smallskip
Exercise: Compare with your neighbour. Are your functions the same? What test assertions could you add to check that behaviour is identical?
\end{frame}

\section{Good practices 1: intermediate}
\subsection{Reduce complexity → easier to test}
\begin{frame}[fragile]{A note on complexity}
We've talked before about cyclomatic (McCabe) complexity:
\begin{minted}{python}
def myfunc(input, option_a: bool, option_b: bool):
    if option_a:
        return input * 1
    else:
        if option_b:
            return input ** 2
        else:
            return input % 3
\end{minted}
\begin{itemize}
    \item What is the complexity of this code?
    \item How many tests (or cases) are required to ensure that the function works as intended?
\end{itemize}

\medskip
Instead of struggling to write many tests for complex code, first \structure{make the code less complex} (for instance, break pieces out into separate functions).

\hspace{14mm}Then: test the less-complex pieces, which will be easier.
\end{frame}

\subsection{Use fixtures and utilities}
\begin{frame}[fragile]{Use fixtures and utilities}
The \href{https://docs.pytest.org/en/stable/how-to/fixtures.html}{pytest fixture HOWTO} is great.
Read, bookmark, and re-read it!

\medskip
\begin{columns}[T]
\column{0.45\paperwidth}
A test often has this structure:

\medskip
\begin{minted}{python}
def test_orig_and_half():
    # A: Set up inputs and outputs
    ...
    # B: Function runs
    result = orig_and_half(input)
    # C: Assertions about `result`
    ...
\end{minted}

\column{0.45\paperwidth}
If “A" and “C" are dozens of lines, then the test is not a clear expression of what \mintinline{python}{orig_and_half()} should do.

\medskip
Instead, split "A" to 1 or more fixtures, or \structure{search for/build on} an existing one.
This is useful even if the fixture is used only in 1 test.
\end{columns}

\medskip
Split "C" to test utility functions, or \structure{search for/build on} an existing one.

\medskip
Example: \href{https://pandas.pydata.org/pandas-docs/stable/reference/api/pandas.testing.assert_frame_equal.html}{\mintinline{python}{pandas.testing.assert_frame_equal()}} (click "[source]") is ~200 lines itself—but only 1 line in your test.
\end{frame}

\subsection{Group and name tests}
\begin{frame}[fragile]{Group and name tests}
Pytest follows \href{https://docs.pytest.org/en/7.1.x/explanation/goodpractices.html\#conventions-for-python-test-discovery}{a simple, well-documented scheme} for naming tests.

\begin{columns}[T]
\column{0.45\paperwidth}
Using this scheme and keeping code and tests \structure{sorted alphabetically} makes it easy to find, modify, and run tests.

\medskip
In \mod{message_ix.tools.foo}:
\begin{minted}{python}
class Foo:
    def __init__(self): ...
    def bar(self, input): ...
    def baz(self, **kwargs): ...
\end{minted}

\column{0.45\paperwidth}
In \mod{message_ix.test.tools.test_foo}:
\begin{minted}{python}
class TestFoo:
    @pytest.fixture
    def obj(self):
        """An instance of Foo."""
        return Foo()

    def test_init(self): ...
    def test_bar(self, obj): ...
    def test_baz0(self, obj): ...
    def test_baz1(self, obj): ...
\end{minted}
\end{columns}
\begin{description}
  \item [\mintinline{shell}{pytest -k Foo}] → match “TestFoo" → all tests of the class.
  \item [\mintinline{shell}{pytest -k _baz}] → match tests of \func{.baz} methods of any class.
  \item [\mintinline{shell}{pytest -k 'Foo and _b'}] → match tests of 2+ Foo methods.
\end{description}
\end{frame}

\section{The MESSAGEix stack: test suites and tools for testing}
\begin{frame}[fragile]{General concepts}
Within one Python package:
\begin{itemize}
  \item Functions, classes in \mod{ixmp} are tested in \mod{ixmp.tests}.
  \item \mod{ixmp.testing} provides things needed for \mod{ixmp.tests}:
  \begin{itemize}
    \item Pytest configuration and hooks.
    \item Fixtures specific to \mod{ixmp}.
    \item Functions (assertions, test data setup, etc.) specific to \mod{ixmp}.
  \end{itemize}
\end{itemize}

\medskip
Across packages:
\begin{itemize}
  \item \mod{message_ix} depends on \mod{ixmp}, \structure{\bfseries therefore:}
  \item Same relationships to \mod{message_ix.tests} and \mod{message_ix.testing}.
  \item \mod{message_ix.tests} can use fixtures and utilities from \mod{ixmp.tests}, \mod{ixmp.testing}. (But not vice versa.)
  \item \mod{message_ix.testing} can extend utilities from \mod{ixmp.testing}.
\end{itemize}
\end{frame}

\begin{frame}{Quiz}
Apply this same logic to understand the relationships of:

\begin{itemize}
  \item \mod{message_data} to \mod{message-ix-models}.
  \item \mod{message-ix-models} to \mod{pandas}.
  \item \mod{ixmp} to \mod{genno}.
\end{itemize}
\end{frame}

\begin{frame}[allowframebreaks]{Valid and robust vs. experimental}
Code in \mod{message_data} and other project-specific repos is often:
\begin{itemize}
  \item Experimental, incomplete,
  \item Only known to work (if at all):
  \begin{itemize}
      \item in a certain person's environment and system; including with specific versions of dependencies.
      \item as integrated within certain workflow(s).
      \item on certain input data file(s).
      \item at a certain point in time (months or years ago).
  \end{itemize}
  \item Undocumented as to expected behaviour and the above intended use.
\end{itemize}

\framebreak
As we go \structure{lower} in the stack \textbf{or} make stronger claims that our code is \structure{scientifically valid} and \structure{FAIR}, it is increasingly critical that our code:
\begin{itemize}
    \item works with all declared supported operating systems, Pythons, and versions of dependencies.
    \item works (or errors in a clear, informative way) on all conceivable input data.
    \item is fully documented as to behaviour and any limitations (=methods).
\end{itemize}

\bigskip
Testing helps us move code from the former condition to the latter.

\medskip
Our internal standard for thoroughness of testing varies.
It depends on where along this spectrum we intend that code should be.
\end{frame}

\begin{frame}[plain]

% \hspace{-10mm}
\includegraphics[height=\paperheight]{Flying\_machine\_patent.jpg}

\end{frame}

\begin{frame}[plain]

% \hspace{-10mm}
\includegraphics[height=\paperheight]{CMS\_at\_CERN\_2\_(40461224333).jpg}

\end{frame}

\begin{frame}{\ttfamily genno}
\framesubtitle{\href{https://genno.readthedocs.io/en/latest/api.html\#module-genno.testing}{\ttfamily genno.testing}}

Fixtures:
\begin{description}
  \item [\py{test_data_path}] path to genno's test data. Construct \emph{sub-}paths only in tests.
\end{description}

\medskip
Assertions:
\begin{description}
  \item [\func{assert_qty_{allclose,equal}}] analogous to \func{pandas.testing.assert_series_*}
  \item [\func{assert_units}] ensure that model-building or reporting computations yield units with correct dimensionality.
  \item [\func{assert_logs}] a\structure{context manager} (\py{with assert_logs(...): ...}: code in the block runs but emits certain log messages.
  \item [\func{raises_or_warns}] another context manager.
\end{description}

\end{frame}

\begin{frame}{\ttfamily ixmp}
\framesubtitle{
\href{https://docs.messageix.org/projects/ixmp/en/stable/api.html\#module-ixmp.testing}{\ttfamily ixmp.testing}}

Fixtures:
\begin{description}
  \item [\py{tmp_env}] modify \textbf{IXMP\_DATA} test I/O and config happens in a temp directory → isn't affected by user's config/files.
  \item [\py{test_mp}] empty platform connected to an in-memory (only) database → fast.
\end{description}

\medskip
Utilities:
\begin{description}
  \item [\py{make_dantzig(mp) -> ixmp.Scenario}] set up a plain-\mod{ixmp} (non-MESSAGE) version of the classic Dantzig cannery/transportation LP
  \item [\func{run_notebook}] execute a Jupyter notebook (IPython, IR, or any other kernel) and make assertions about certain cells' outputs.
\end{description}

\medskip
Quiz: How could you use \mod{test_mp} and \func{make_dantzig} together?

\hspace{12mm} What would this provide for your test function?
\end{frame}

\begin{frame}{\ttfamily message\_ix}
\framesubtitle{
\href{https://docs.messageix.org/en/latest/api.html\#module-message_ix.testing}{\ttfamily message\_ix.testing}}

Fixtures:
Few global fixtures; but heavy use of fixtures from \mod{ixmp.testing}.
\begin{description}
  \item [\py{message_ix_cli}] uses the \py{click.CliRunner} test utility in a temporary environment.
  Allows to invoke the \ttfamily{message-ix} command-line interface (for instance, \ttfamily{message-ix show-versions}), capture result and exit code, and make assertions about these.
\end{description}

\medskip
Utilities:
\begin{description}
  \item [\py{make_{austria,dantzig,westeros}() -> message_ix.Scenario}] make and optionally solve these well-defined scenarios.
\end{description}

\medskip
Examples:
\begin{enumerate}
  \item \href{https://github.com/iiasa/message_ix/blob/a488b8a/message_ix/tests/test_report.py\#L103-L146}{\func{message_ix.tests.test_report.test_reporter_from_westeros}}
  \item \href{https://github.com/iiasa/message_ix/blob/a488b8a87ac34162da05577b0edd5f1012a33d1a/message_ix/tests/test_cli.py\#L11-L50}{\func{message_ix.tests.test_cli.test_copy_model}}
\end{enumerate}
\end{frame}

\begin{frame}{\ttfamily message-ix-models}
\framesubtitle{\href{https://docs.messageix.org/projects/models/en/latest/api/testing.html}{\ttfamily message\_ix\_models.testing}}

Fixtures:
\begin{description}
  \item [\py{test_context}] a temporary instance of the \href{https://docs.messageix.org/projects/models/en/latest/api/util.html\#message_ix_models.util.context.Context}{\py{Context}} class; scoped to an individual test function.
  \item [\py{mix_models_cli}] another CliRunner. Since all \mod{message_data} CLI groups and commands are attached to message-ix-models, this can also be used there.
\end{description}

\medskip
Utilities
\begin{description}
  \item [\py{bare_res(request, context, ...) -> message_ix.Scenario}] a scenario with \emph{all of the structure} of a MESSAGEix-GLOBIOM scenario (according to \py{context}), but empty of data.
\end{description}
\end{frame}

\begin{frame}{\ttfamily message\_data}

\structure{\bfseries No \mod{.testing} module}.
\begin{itemize}
  \item The \emph{scope and role} of \mod{message_data} is the same as \mod{message-ix-models}; the only difference is the former contains code/data that is not yet (or can never be) made public.
  \item Therefore, anything generalizable across all of \mod{message_data} is also generalizable across all of \mod{message-ix-models}, and should go in \mod{message-ix-models.testing}.
\end{itemize}

\bigskip
Code for specific model variants and projects may contain testing submodules.
Example: \href{https://docs.messageix.org/projects/models-internal/en/latest/reference/model/_autosummary/message_data.model.transport.testing.html}{\mod{message_data.model.transport.testing}}

\bigskip
When these modules are later migrated to \mod{message-ix-models}, the testing submodule is carried along.
\end{frame}

\section{Good practices 2: intermediate/advanced}
\subsection{Parametrize}
\begin{frame}{Parametrize}
\framesubtitle{\mintinline{python}{@pytest.mark.parametrize()}}

\begin{itemize}
  \item Give 1 or more arguments to a test function, from a Python sequence.
  \item These can be \structure{inputs or outputs} for the target function.
  \item Ask: “What are all the ways we might use this function?”
  \item “What mistakes might I make in using this function?”
\end{itemize}

\medskip
Examples:
\begin{enumerate}
  \item tests of
\href{https://github.com/iiasa/message_ix/blob/a488b8a/message_ix/tests/test_core.py\#L112-L205}{\func{message_ix.Scenario.add_horizon}}
  \item \href{https://github.com/iiasa/message-ix-models/blob/d985e5d/message_ix_models/tests/project/test_ssp.py\#L134-L158}{\func{message_ix_models.….TestSSPUpdate.test_prepare_computer}} —use parametrize 2+ times → Cartesian product.
\end{enumerate}

\end{frame}

\subsection{Use xfail and other marks}
\begin{frame}{Use xfail and other marks}
\framesubtitle{\mintinline{python}{@pytest.mark.xfail(raises=ValueError, reason="Incomplete")}}

\begin{itemize}
  \item Also \mintinline{python}{@pytest.mark.skipif()}, others.
  \item Again, great docs: \url{https://docs.pytest.org/en/stable/how-to/mark.html}
  \item Write a test that will not pass yet—e.g. an integration test of 2+ functions, when you have not written either.
  \item Temporarily \& explicitly confirm certain tests will fail while code is refactored.
  \item Group tests to be (de)selected with the \textbf{pytest -m} command-line option.
\end{itemize}

\bigskip
Examples:
\begin{enumerate}
\item \href{https://github.com/iiasa/message_ix/blob/a488b8a/message_ix/tests/test_core.py\#L112-L186}{\func{message_ix.tests.test_core.test_add_horizon}} —combine parametrization and marks.
\end{enumerate}
\end{frame}

\subsection{Isolate/avoid leakage}
\begin{frame}[fragile]{Isolate/avoid leakage}
\begin{columns}[T]
\column{0.45\paperwidth}
\begin{minted}{python}
from pathlib import Path

def test_a():
    p = Path("file.txt")
    assert not p.exists()
    p.write_text("Hello, world!")
    t = p.read_text()
    assert t.endswith("!")
\end{minted}

\column{0.45\paperwidth}
\begin{minted}{python}
from pathlib import Path

def test_b():
    p = Path("file.txt")
    assert not p.exists()
    p.write_text("How are you?")
    t = p.read_text()
    assert t.endswith("?")
\end{minted}
\end{columns}

\medskip
How can these tests be troublesome?\pause
\begin{itemize}
  \item If run in sequence, one test will find the file made by the other and fail.
  \item If run in parallel (using \href{https://pytest-xdist.readthedocs.io/en/latest/}{pytest-xdist} as we do) one test could have "file.txt" open for writing at the time the other tries to open/write it.
\end{itemize}
\end{frame}

\begin{frame}[fragile]{Isolate/avoid leakage II}
As a general rule: tests should not alter anything external to them—the user's file system, environment variables, global settings, etc.

\medskip
Many ways to avoid this, mainly by using fixtures: from pytest, third-party packages, ours, or your own.
\begin{columns}[T]
\column{0.45\paperwidth}
\begin{minted}{python}
from pathlib import Path

def test_a(tmp_path):
    # File in a per-test
    # temporary directory
    p = tmp_path.joinpath("file.txt")
    assert not p.exists()
    p.write_text("Hello, world!")
    t = p.read_text()
    assert t.endswith("!")
\end{minted}

\column{0.45\paperwidth}
\begin{itemize}
  \item \href{https://docs.pytest.org/en/stable/reference/reference.html\#pytest.FixtureRequest.node}{\ttfamily request} fixture and \mintinline{python}{request.node.name} to get a unique string like "test\_foo.py::test\_bar[case1]".
  \item \href{https://docs.pytest.org/en/stable/how-to/monkeypatch.html\#monkeypatching}{\ttfamily monkeypatch} to modify attributes or elements of global objects—then restore the original values, regardless of whether a test passes or fails.
\end{itemize}

\end{columns}
\end{frame}

\begin{frame}[fragile]{Isolation example}
\framesubtitle{\href{https://github.com/iiasa/ixmp/blob/049689a/ixmp/tests/core/test_scenario.py\#L22-L30}{\mod{ixmp.tests.core.test_scenario.scen_empty} fixture}}

\py{request} is a Pytest built-in fixture. \py{request.node.name} identifies 'node'—the particular test function \emph{and} parametrization of it—as a string, for instance:
\begin{minted}{text}
ixmp/tests/backend/test_jdbc.py::TestJDBCBackend::test_init_item[COST_NODAL_NET-idx_sets2]
ixmp/tests/backend/test_jdbc.py::TestJDBCBackend::test_init_item[DEMAND-idx_sets3]
\end{minted}

\medskip
The node name is guaranteed unique for the entire package.

\medskip
\mintinline{python}{ixmp.Scenario(..., model=request.node.name, ...)} this will \emph{never} be generated by another request for this \py{scen_empty} fixture.
\end{frame}

\subsection{Move toward fuzzing}
\begin{frame}{Move toward ‘fuzzing’}
Formally, \href{https://en.wikipedia.org/wiki/Fuzzing}{‘fuzzing’} includes a range of practices such as giving “aggressively incorrect” inputs to code to ensure it is robust against malicious use.

\medskip
For our purposes, most useful to consider a hierarchy of quality:
\begin{enumerate}
    \item Code works when run on certain input data, in a specific workflow that you run in your environment (“experimental”).
    \item Code works when run by a colleague (outside of your environment).
    \item Code works \structure{within a test/test suite} (outside of the workflow).
    \item Code works when run on \structure{different input data} (not the data from your workflow) ($N=2$).
    \item Code works when run on \structure{generated input data} ($N \gg 2$).
    \item Code works when run on  \structure{\textbf{‘fuzzed’} (random) input data} ($N \rightarrow \infty$).
\end{enumerate}

\medskip
\hspace{14mm}→ Moving from (1–2) to (3–6) makes methods genuinely reusable.
\end{frame}

\subsection{Split unit and integration tests}
\begin{frame}{Split unit and integration tests}

Previous slide (1), if written down, is a kind of \emph{\textbf{manual} integration test}: “run such-and-such command or script, get a result with such-and-such characteristics”.

\smallskip
A test (in pytest, or a continuous integration/GitHub Actions workflow) that \emph{automates} this is an automated integration test.

\medskip
\begin{itemize}
  \item Any one tests is \structure{either an \textbf{integration} or \textbf{unit}} test.
  \item Code that has only one kind \structure{does not have} the other.
  \item Code \structure{should have both} kinds of tests.
\end{itemize}

\medskip
If integrated systems are large and complex, \structure{intermediate-scope} integration tests are also useful.
More thorough (e.g. fuzzed) testing of smaller code units also supports generalizing the larger system.
\end{frame}

\section{Hands-on: write or improve a test}
\begin{frame}[fragile]{Hands-on: write/improve a test}

\begin{enumerate}
  \item Choose a piece of code you wrote or use—perhaps one where the behaviour has been perplexing.
  \item Locate existing tests of that code (if any).
  \item Improve the tests:
  \begin{itemize}
    \item Add missing tests.
    \item Add parametrized cases to an existing test.
    \item Simplify overly long/complicated tests.
    \item Collect duplicated code in a fixture or utility function.
  \end{itemize}
\end{enumerate}

\medskip
We will stop at $3N+5$ minutes to the hour (N is the number of participants).
Each can show a \verb@git diff@ of their improvements.

\smallskip
Please ask questions and discuss with your neighbours.
\end{frame}

\section*{Thank you!}

\begin{frame}{Credits}

Fridolin Glatter —comments and feedback

\medskip
Images:
\begin{itemize}
  \item \href{https://commons.wikimedia.org/wiki/File:CMS_at_CERN_2_(40461224333).jpg}{CMS\_at\_CERN\_2\_(40461224333).jpg} —Ryan Bodenstein via Wikimedia Commons.
  \item \href{https://commons.wikimedia.org/wiki/File:Flying_machine_patent.jpg}{Flying\_machine\_patent.jpg} —public domain, via Wikimedia Commons.
\end{itemize}
\end{frame}

\end{document}
