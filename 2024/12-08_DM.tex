\documentclass[a5paper,11pt]{article}

\usepackage{amsmath}
\usepackage{amssymb}
\usepackage{enumitem}
  \setlist{nosep}
  \setlist[itemize]{label=---}
\usepackage[hmargin=10mm,vmargin=15mm]{geometry}
\usepackage{graphviz}
\usepackage[colorlinks]{hyperref}

\title{Quantifying decent mobility: \\
  Concepts, methdology, and proposals}

\author{P.N. Kishimoto for the EDITS Decent Mobility FT2024}

\setcounter{secnumdepth}{0}

\newcommand{\DLS}{\text{DLS}}
\newcommand{\DM}{\text{DM}}

\begin{document}
\maketitle
\tableofcontents

\section {0: Simplest case}

We ask every individual ($i$) in the entire population of the world ($I$), some question like “Are you well? Do you have wellbeing?” and encode their answers as \textbf{data} $x^0_i$.
In the simplest possible case, these are \emph{boolean} values: ‘no’/‘false’/0 or ‘yes’/‘true’/1.

The \textbf{metric} for \emph{overall\footnote{Some nuance on this later.} wellbeing} $W^0$ is expressed as a particular function that can be computed on the data.
The simplest possible case is \emph{logical conjunction} (‘and’/ $\land$) of all $x^0$.
This takes the value of 1/‘true’/‘yes’ iff all $x^0_i$ have that value.
In other words, overall or universal wellbeing exists if—and only if—person 1 has wellbeing \emph{and} person 2 has wellbeing \emph{and} so on.%
\footnote{Although this may seem simple and reductive, it actually corresponds to real policy discourse.
For instance, the phrase “No one is safe until everyone is safe” (used for instance \href{https://www.unicef.org/press-releases/no-one-safe-until-everyone-safe-why-we-need-global-response-covid-19}{here}, by UNICEF) has about the same meaning as this metric: if $x^0_i=0$ for any $i$, then $W^0(x^0) = 0$.}

\begin{align}
      i & \in I = \{i_1, i_2, \ldots\} \\
    |I| & = 8,000,000,000 \\
  x^0_i & \in \mathbb{B} = \{0, 1\} \\
    W^0(x^0) & = \bigwedge_{i \in I}{x^0_i} \in \mathbb{B} \label{eq:w0}
\end{align}

All of the following cases are distinguished by different data, $x^{\cdots}$, or functions, $W^{\cdots}$.
(A right superscript like $\cdots^0$ is an index.
$x^1$ is a different measurable quantity than $x^0$,
$W^1$ is a different function or metric than $W^0$, etc.)

\section{1: Thresholds; same data \& different metric}

Define a different metric, $W^1$: the fraction of all individuals that have wellbeing.
With a metric like this one could say, “Overall wellbeing is obtained if $W^1$ is greater than some \textbf{threshold value}, $W^{\ast 1}$”:

\begin{align}
  W^1(x^0) & = \frac{\sum_{i \in I}{x^0_i}}{|I|} \\
  W^1 & \geq W^{\ast 1} = 0.8
\end{align}

This case shows it is possible that $W^1(x^0) \geq W^{\ast 1}$ while at the same time $W^0(x^0) = 0$.
That is: with \emph{the same data}, different metrics give a different verdict on whether collective wellbeing is achieved.
The word ‘universal’ would seem to imply something more like $W^0$, but often words like ‘overall’, ‘universal’ etc. are used describe things that are more like $W^1$.

\section{2: Derived data; same metric \& different data}

In practice, we can't go and ask everyone on the planet “Are you well?”
Suppose we could instead \emph{observe} something \emph{about} every individual $i$—for example their income $x^1_i$—and choose to use this data.%
\footnote{Of course it is also not practical to observe any thing about every individual.}
Then we can \emph{derive} further or intermediate data \eqref{eq:x2} and compute \emph{the same metric} as above \eqref{eq:w0} on these different data:%
\footnote{We use the term \textbf{functions} to refer to both metrics and functions used to compute intermediate data.}

\begin{align}
  x^1_{i}         & \geq 0 \text{ dollars / day} & \in \mathbb{R} \\
  x^2_{i}(x^1_i)  & = \begin{cases}
                        0 & x^1_i < x^{\ast 1} = 2 \\
                        1 & x^1_i \geq x^{\ast 1}
                      \end{cases}
                  & \in \mathbb{B} \label{eq:x2} \\
  W^0(x^2)        & = \bigwedge_{i \in I}{x^2_i}
\end{align}

Similar to the above case, it is possible that $W^0(x^2) = 1$ while $W^0(x^0) = 0$.
If there is 1 individual among the 8 billion in $I$ whose income is more than 2 dollars per day, but answers (or would answer) the question “Are you well?” with ‘no’, then this will be the outcome.

Some further points from this case:
\begin{itemize}
  \item The threshold here $x^{\ast 1}$ pertains to the data, or individual observations, in contrast to Case 1 where $W^{\ast 1}$ was a threshold for the metric.
  \item In both cases the threshold is a single, scalar value.
    In practice, different thresholds are often used for different $i$.
  \item The function $x^2_i(x^1_i)$ is a very simple example of what economists call a \emph{utility function} or we might call a \emph{model}.
    We are in effect saying: “We \emph{model} (suppose, judge) that if we know individual $i$ has income more than 2 dollars per day, they would say ‘Yes, I am well.’”

    This makes clear that $x^2_i$, or any other kind of modeled, derived, or imputed wellbeing, is not the same as $x^0_i$.
    The latter is (would be) actual data from an individual; the former is what we think their answer could or should be.
\end{itemize}

\section{3: Representative agents; sub-populations}
Methodologically, ‘representativeness’ means to use data like $\bar{x}^{\cdots}$ that are some statistic (the over-bar denotes the arithmetic mean, but others such as medians etc. are also used) computed on data $x^{\cdots}$ for entire populations $I$ or sub-populations.
These are used together with functions that yield values that are estimates ($\hat{\cdots}$) of statistics across entire populations.

\begin{align}
                   \bar{x}^{\cdots} & = \frac{\sum_{i \in I}{x^{\cdots}_i}}{|I|} \\
                   \bar{W}^3 & = \frac{\sum_{i \in I}{W^3(x^{\cdots}_i)}}{|I|} \\
  \hat{W}^3 = W^3(\bar{x}^{\cdots}) & \sim \bar{W}^3
\end{align}

For models and analyses that split the world into geographic areas or regions $r \in R = \{r_1, r_2, \ldots\}$, there is often one sub-population $I_r$ for each area.
Using a “representative agent” for each area means to compute the same intermediate data functions or metrics on data which are the mean of values for each individual the sub-population:
\begin{align}
                       \bar{x}^{\cdots}_r & = \frac{\sum_{i \in I_r}{x^{\cdots}_i}}{|I_r|} \\
  \hat{W}^3_r = W^3(\bar{x}^{\cdots}_r) & \sim \bar{W}^3_r
\end{align}

The outcomes are similarly estimates for that sub-population.
To re-obtain $\hat{W}^A$—that is, the estimate of a statistic for the total population $I$—it is necessary to \emph{aggregate} $\hat{W}^A_r$ by an appropriate calculation such as summation or (weighted) average.

In \emph{agent based modeling (ABM)}, a total population is segmented not (or not only) by geographic area or location, but according to some methods or other data.%
\footnote{For instance, the total population of each region might be split two sub-populations of “people who commute” and “people who do not commute”; or using clustering methods that operate on other data.}
These can either be represented hierarchically as some sub-populations or groups $g \in G_r$, or not in which case the groups $G$ also separate individuals according to geography.
In the former case, \emph{intermediate aggregates} are often computed or used.

\section {4: Decent mobility vs. wellbeing}
The simplest relationship between these two concepts—and one consistent with our grounding in the theory of needs and satisfiers—is that decent mobility is necessary but not sufficient for wellbeing.
Thus:
\begin{itemize}
  \item If an individual has decent mobility, they \emph{may} have wellbeing.
  \item If an individual does not have decent mobility, they also do not have wellbeing.
  \item If an individual has wellbeing, they \emph{must} have decent mobility.
  \item If an individual does not have wellbeing, it is indeterminate whether they have decent mobility, or not.
\end{itemize}

Mathematically:
\begin{align}
  \hat{x}^0_i & = f(x^{\DM}, \cdots) \label{eq:xDM}
\end{align}

\section {Goals}

In the original DLS work:
\begin{itemize}
  \item $x^{\DLS}$ was expressed as “travel distance in a year by all modes of motorized transport.”
  \item A threhold of 10,000 km was used for $x^{\ast\DLS}$.
  \item $W^{\DLS}$ seems to have been one of:
    \begin{align}
      \hat{W}^{\DLS} = W^{\DLS}(\bar{x}^{\DLS}) & = \begin{cases}
        0 & \bar{x}^{\DLS} < x^{\ast\DLS} \\
        1 & \bar{x}^{\DLS} \geq x^{\ast\DLS}
      \end{cases} \\
      \hat{W}^{\DLS} = W^{\DLS}(\bar{x}^{\DLS}) & = \begin{cases}
        0 & f(\bar{x}^{\DLS}, \cdots) < x^{\ast\DLS} \\
        1 & f(\bar{x}^{\DLS}, \cdots) \geq x^{\ast\DLS}
      \end{cases}
    \end{align}
    …for instance $f(x, \cdots)$ being computed with Gini or other additional data.
\end{itemize}

\noindent Our goals with this project:
\begin{enumerate}
  \item Choose some \emph{measurable quantities} $x^{\DM0}$, $x^{\DM1}$, etc.
  \item Possibly choose or define thresholds ($\cdots^\ast$) for these.
  \item Choose some \emph{derived measures} $x^{\DM2}$, $x^{\DM3}$, etc.
  \item Possibly choose or define thresholds for these.
  \item Choose or define metric(s) ($W^{\DM}$).
  \item Possibly choose or define thresholds for these.
\end{enumerate}

\section {Proposals}

\begin{enumerate}[label=P\arabic*]
  \item We define “(universal) decent mobility” using data and a metric similar to \eqref{eq:w0}:
    \begin{align}
      x^{\DM0}_i       & = \begin{cases}
                             0 & \text{individual } i \text{ does not have decent mobility} \\
                             1 & \text{individual } i \text{ has decent mobility}
                           \end{cases} \\
      W^{\DM}(x^\DM_i) & = \bigwedge_{i \in I}{x^{\DM0}_i} \in \mathbb{B} \label{eq:wDM}
    \end{align}
    This has several strengths:
    \begin{itemize}
      \item It makes clear that the wellbeing/decent mobility \emph{of individuals} is fundamental.
        Work (including ours) that does not deal directly with individuals—for instance, using statistics that make possible that some individuals may not have decent mobility—must acknowledge and defend these departures.
      \item The relationships \eqref{eq:xDM} and $W^0 = f(W^{\DM}, \cdots)$ are simple, direct, clear.
    \end{itemize}
  \item We frame our quantitative work as attempts to \emph{estimate} these quantities, or various intermediate sub-aggregations.
  \item We define additional metrics that help quantify \emph{decent mobility gaps}, or the extent to which decent mobility is \emph{not} achieved.
    For example:
    \begin{align}
      W^{gap 1}(x^{\DM0}) & = 1 - \frac{\sum_{i \in I}{x^{\DM}_i}}{|I|} \\
      W^{gap 2}(x^{\DM \cdots}) & \sum_{i \in I, k \in \{1, 2, \ldots\}}{\min(x^{\DM k}_i - x^{\ast \DM k}, 0)}
    \end{align}
  \item We identify \textbf{base data} $x^{\DM \cdots}$ including:
    \begin{itemize}
      \item Measures of the local context faced by an individual.
      \item Measures of individuals' attributes.
      \item Measures of trips that are (a) possibly in the individual's context, and (b) accessible to the individual in particular.
    \end{itemize}
    …these measures can include things like distance, duration, cost, energy use, etc. of trips, trip segments, tours, travel plans, etc.
  \item We express the various \emph{constraints} in our conceptual framework as \textbf{intermediate/derived data} and accompanying \textbf{thresholds}.
    For example: we define $x^{\text{TT}}_i$ as “total travel time [hours] for an individual's accessible trips in one day” and set a maximum threshold $x^{\ast \text{TT}} = 1.2$.
    Then:
    \begin{align}
      \hat{x}^{\DM0}_i =
    \left( x^{\text{TT}}_i \leq x^{\ast \text{TT}} \right) \land \cdots
    \end{align}
    …in other words, the given intermediate data being below the given threshold is \emph{one necessary condition} for the individual to be estimated to have decent mobility.
  \item We clarify that our \textbf{personas} are representative agents, but for particular sub-populations that are likely to not have universal decent mobility.
  Thus they do not represent entire populations in certain areas; nor do we give a large set of personas that \emph{collectively} represent all the population of a given area.
  We also distinguish ourselves from the ABM approach of choosing the \emph{fewest} representative agents that represent the \emph{largest} or \emph{highest-consuming} portions of a total population.
\end{enumerate}

% \digraph{abc}{
%   rankdir=LR;
%   a -> b -> c;
% }

\end{document}
