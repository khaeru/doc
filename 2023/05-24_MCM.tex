% NB title page only; attached to the 2021-12-01 slide deck mentioned

\documentclass[12pt,aspectratio=169]{beamer}
\usetheme{iiasa}

% % Align with source slides
% % Imitate the 'metropolis' theme
% \setsansfont[
%   Extension = .ttf,%
%   UprightFont = FiraSans-Regular,%
%   ItalicFont = FiraSans-Italic,%
%   BoldFont = FiraSans-Bold,%
%   BoldItalicFont = FiraSans-BoldItalic,%
%   ]{FiraSans}%
% \setmonofont[
%   Extension = .ttf,%
%   UprightFont = FiraMono-Regular,%
%   BoldFont = FiraMono-Medium]%
%   {FiraMono}%
% \setsansfont{FiraSans}
%
% % Since the default weight is light, make titles bold
% \setbeamerfont{title}{size=\Huge, series=\bfseries}
% \setbeamerfont{frametitle}{size=\LARGE, series=\bfseries}
% \setbeamerfont{section title}{size=\Huge, series=\bfseries}

\usepackage{ifthen}
\ifthenelse{\equal{\detokenize{notes}}{\jobname}}{%
\setbeameroption{show only notes}%
}{
%
}

\usepackage[
  maxnames = 1,
  style = authoryear,
  giveninits,
  terseinits,
  maxcitenames = 3,
  ]{biblatex}
\addbibresource{all.bib}
% Small font in bibliography
\renewcommand*\bibfont{\small}
\addbibresource{all.bib}

\usepackage{minted}
\usepackage{pdfpages}

\usepackage{tikz}
\usetikzlibrary{calc}

\title{MESSAGE\emph{ix}-Transport}
\subtitle{Introduction and overview}
\institute{Energy, Climate, and Environment (ECE) Program \\
  International Institute for Applied Systems Analysis (IIASA)}

\date{\texorpdfstring{Message Community Meeting — Wednesday, 24 May 2023
  {\scriptsize \\ \emph{incl. material from} 14th IAMC Annual Meeting, 2021-12-01 \emph{w/} Lovat, van Ruijven, Krey, Riahi}}%
  {2023-05-24}}

\author{\texorpdfstring{Dr. Paul Natsuo Kishimoto \scriptsize\newline
  \href{mailto:paul.kishimoto@iiasa.ac.at}%
       {\ttfamily <paul.kishimoto@iiasa.ac.at>}}%
  {Dr. Paul Natsuo Kishimoto <paul.kishimoto@iiasa.ac.at>}}

\begin{document}

\maketitle

\begin{frame}
  \frametitle{Contents}
  \tableofcontents
\end{frame}

\section{MESSAGEix-Transport}

\subsection{Introduction, goals}
\subsection{Model structure, data flows}
\subsection{Core methods}
\subsection{Input data, methods for preparation}

\includepdf[pages={4-17}]{../2021/12-01_IAMC.pdf}

\section{Planned improvements}

\subsection{Initial public release}

\begin{frame}
  \frametitle{Initial public release}

  Work beginning 2023-Q3, completing 2023-Q4.

  \bigskip
  Target:
  \begin{itemize}
    \item \texttt{message\_ix\_models.model.transport} —i.e. a module in the public MESSAGEix modeling tools repository, next to \texttt{.water} and others.
    \item Complete documentation \& test suite.
    \item Data prepped for R12 nodes, with clear declaration of data needs for application to other geographies.
  \end{itemize}
\end{frame}

\subsection{Shared \& autonomous LDV techs}
\subsection{Operational concept of ‘activity’}

% Planned improvements
\includepdf[pages={31-32}]{../2021/12-01_IAMC.pdf}

\section{Energy services in general}

\begin{frame}
  \frametitle{Energy services in general}

  \tableofcontents[currentsection]

\end{frame}

\subsection{MESSAGE/LP techniques for demand-side sectors}

\begin{frame}[allowframebreaks]
  \frametitle{LP techniques for demand-side sectors}

  \bigskip
  Original/natural mapping of the MESSAGEix \texttt{commodity}, and \texttt{technology} concepts:
  \begin{itemize}
    \item A measurable quantity of a physical object, e.g. \texttt{coal}.
    \item A artifact/machine that performs a physical process, e.g. \texttt{coal\_ppl}.
  \end{itemize}

  \bigskip
  \emph{Energy services} are often conceived as intangible (“health”, “access”), socio-cultural concepts measured as constructed quantities (“disability-adjusted life-year (DALY)” vs. “kg of coal”).

  \framebreak
  \structure{“Pseudo”}-technologies and commodities.
  \begin{itemize}
    \item Represent a quantity that is \emph{measurable} but perhaps not \emph{physical} as a “\texttt{commodity}”.
    \item Express MESSAGEix \texttt{demand} in these commodities.
    \item Represent a socio-technical process or relationship between two commodities (in the above sense) as a “\texttt{technology}”.
  \end{itemize}

  \bigskip
  MESSAGEix-Transport example:
  \begin{itemize}
    \item Transport activity various measures (distance traveled by a vehicle; distance traveled by a person; trips) → \texttt{commodity}
    \item “Usage” and “occupancy” of vehicles by people to obtain the energy service of passenger-distance traveled → \texttt{technology}.
  \end{itemize}

\end{frame}

\subsection{Systems theory for global modeling}

\begin{frame}[allowframebreaks]
  \frametitle{Systems theory for global modeling}

  In general, IAMs are always dependent on data and other models that treat the real-world phenomena in much greater detail, with domain-specific theory \& methods.

  \bigskip
  For example, in transport:
  \begin{itemize}
    \item Short-distance passenger: agent-based microsimulation of daily activity particular cities or model cities.
    \item Interaction of land-use, built environment, and mobility: “ILUTE”.
    \item Choice of vehicle technology or new modes (sharing, etc.):
    \item Aviation: models of airport expansion, new aircraft technologies.
    \item International freight: gravity models connecting ports.
  \end{itemize}

  \bigskip
  How do we preserve the information/validity of these while connecting or emulating them in IAMs?

  \bigskip
  → Recall and apply engineering systems theory.
\end{frame}

% Systems slides
\includepdf[pages={35-44}]{../2021/12-01_IAMC.pdf}

\begin{frame}[plain]

  \centering \Huge \structure{\bfseries Thank you!}

\end{frame}

\begin{frame}
  \frametitle{Contents}
  \tableofcontents
\end{frame}

% References
\includepdf[pages={28}]{../2021/12-01_IAMC.pdf}

\end{document}
