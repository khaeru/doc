\documentclass[12pt,aspectratio=169]{beamer}
\usetheme{iiasa}

\usepackage{ifthen}
\ifthenelse{\equal{\detokenize{notes}}{\jobname}}{%
\setbeameroption{show only notes}%
}{
%
}

\usepackage[
  maxnames = 1,
  style = authoryear,
  giveninits,
  terseinits,
  maxcitenames = 3,
  ]{biblatex}
\addbibresource{all.bib}
% Small font in bibliography
\renewcommand*\bibfont{\small}
\addbibresource{all.bib}

\usepackage{minted}
\usepackage{pdfpages}

\usepackage{tikz}
\usetikzlibrary{calc}

\title{Developing open energy models}
\subtitle{Forum on practices and experiences}
\institute{Energy, Climate, and Environment (ECE) Program \\
  International Institute for Applied Systems Analysis (IIASA)}

\date{\texorpdfstring{Openmod Workshop\\
  Wednesday, 22 March 2023}%
  {2023-03-22}}

\author{\texorpdfstring{Dr. Paul Natsuo Kishimoto \scriptsize\newline
  \href{mailto:paul.kishimoto@iiasa.ac.at}%
       {\ttfamily <paul.kishimoto@iiasa.ac.at>}}%
  {Dr. Paul Natsuo Kishimoto <paul.kishimoto@iiasa.ac.at>}}

\begin{document}

\maketitle

\begin{frame}
\frametitle{Outline}

\tableofcontents

\end{frame}

\section{Introduction}

\begin{frame}
\frametitle{Introduction, session goals}

A \structure{fully interactive} discussion on model-building practices.
\begin{itemize}
  \item Examples from MESSAGE\emph{ix} to prompt discussion on each topic.
  \item Sharing from participants (please introduce yourself).
\end{itemize}

\bigskip
Goals:
\begin{itemize}
  \item Learn from one another how to effectively use available tools…
  \item …to efficiently create high-quality tools, models, research.
  \item Identify topics for further study or bi-/multi-lateral exchange.
\end{itemize}

\end{frame}

\subsection{General considerations \& session goals}

\begin{frame}
\frametitle{General considerations}

We all engage in these activities:
\begin{enumerate}
  \item Develop (extend, maintain, …) modelling tools.
  \item Use those tools to create specific models.
  \item Use those specific models to conduct model-based research.
    \begin{itemize}
      \item Produce research products: documents, data, figures, …
    \end{itemize}
  \item Support others/a community in using the tools, models, or products.
\end{enumerate}

\medskip
These compete for \structure{time and resources} and can lead to tension in priorities.
This can lead to choices such as:
\begin{itemize}
  \item Increase complexity of a specific model \\
    \structure{\bfseries or} improve validation of existing tools.
  \item Write developer- or user-focused documentation \\
    \structure{\bfseries or} polish research products for a scientific or policy audience.
\end{itemize}

\end{frame}

\begin{frame}
\frametitle{General considerations (cont'd)}

Activities, goals, and resources (time, staff, skills) naturally vary across people \& organizations.

\bigskip
Thus, in your interventions:
\begin{itemize}
  \item Briefly state (or recap) your affiliation, context, and goals.
  \item Say \emph{why} a particular practice has made sense in that context.
\end{itemize}
…this helps others understand if their context is similar enough to adopt your practices.

\bigskip
Info on what \structure{\bfseries has not} worked can be as or more instructive than success stories.

\end{frame}

\subsection{MESSAGE\emph{ix} development context}

\begin{frame}
\frametitle{MESSAGE\emph{ix} development context}

Users:
\begin{itemize}
  \item Largest user category is in-house: about 20 in the team.
  \item A spectrum from user/developer/maintainer to ‘pure’ user.
  \item Most outside users are collaborators with in-house users → conduit for support.
\end{itemize}

\bigskip
Key objectives:
  \begin{itemize}
    \item Publications in high-impact journals.
    \item Contributions to integrated assessment modeling (IAM) community activities \& consortium research.
    \item Advance IIASA's open science/outreach goals.
  \end{itemize}

\end{frame}

\begin{frame}
\frametitle{MESSAGE\emph{ix} development context}

\begin{description}
  \item [Version control] Git.
  \item [Collaboration] GitHub (\href{https://github.com/iiasa}{github.com/iiasa} org), Slack (paid).
  \item [Language] Python, some R and Java.
  \item [CI] GitHub Actions, ReadTheDocs (paid), Codecov.
\end{description}

\bigskip
\begin{itemize}
  \item Popular/common tools and platforms → greater chance that incoming colleagues have experience.
\end{itemize}

\end{frame}

\section{Specific topics}

\begin{frame}
\frametitle{Specific topics}

\tableofcontents[currentsection,hideothersubsections]

\end{frame}

\subsection{Modularize, separate concerns}
\begin{frame}[allowframebreaks]
\frametitle{Modularize, separate concerns}

A “stack” of 5 packages:
\begin{enumerate}
  \item [0] \texttt{iiasa/ixmp\_source} (private) Java/Oracle backend for \texttt{ixmp}.
  \item \href{https://github.com/iiasa/ixmp}{\ttfamily iiasa/ixmp} data model \& storage; GAMS interface for generic models.
  \item \href{https://github.com/iiasa/message_ix}{\ttfamily iiasa/message\_ix} generic MESSAGE in GAMS; utilities.
  \item \href{https://github.com/iiasa/message-ix-models}{\ttfamily iiasa/message-ix-models} data etc. for the MESSAGE\emph{ix}-GLOBIOM family of global \& single-country models.
  \item \href{https://github.com/iiasa/message_data}{\ttfamily iiasa/message\_data} (private) unpublished/in-progress projects; tools for proprietary data.
\end{enumerate}
\medskip
…plus
\href{https://github.com/IAMConsortium/iam-units}{\ttfamily IAMConsortium/iam-units},
\href{https://github.com/khaeru/genno}{\ttfamily khaeru/genno},
\href{https://github.com/IAMConsortium/pyam}{\ttfamily IAMConsortium/pyam},
\texttt{message\_single\_country} (private)
others

\framebreak
Practices and their benefits:
\begin{itemize}
  \item Progression of stability/testing standards.
  \item API changes in lower levels → clear signal to adjust higher levels.
  \item Spot and avoid “stack violations”: low-level code being overloaded to meet needs of higher level code.
  \item Genericize/improve case-specific functions and tools to general-purpose utilities.
\end{itemize}

\end{frame}

\subsection{Split private/public code \& data}

\begin{frame}
\frametitle{Split private/public code \& data}

\href{https://github.com/iiasa/message-ix-models}{\ttfamily iiasa/message-ix-models} vs.
\href{https://github.com/iiasa/message_data}{\ttfamily iiasa/message\_data}.

\medskip
Legitimate reasons for non-public code:
\begin{itemize}
  \item Enhancements to models or project-specific innovations that must be embargoed until published.
  \item Prototype/attic code, no resources to support outside users in a public package.
  \item Data or code for handling proprietary input data.
\end{itemize}

\medskip
Practices:
\begin{itemize}
  \item Contain code in specific Python modules:
    \begin{itemize}
      \item Model representation of certain energy systems phenomena.
      \item Set-up of scenarios/analysis for specific projects.
    \end{itemize}
  \item Same code style, practices, etc.
  \item Guides for moving code, data from private → public (e.g. \href{https://github.com/iiasa/message-ix-models/pull/89}{\#89}).
\end{itemize}

\end{frame}

\subsection{Use continuous testing/integration}
\begin{frame}
\frametitle{Use continuous testing/integration}

Cloud-based systems that automatically run jobs, composed of steps.
\begin{itemize}
  \item GitHub Actions and ReadTheDocs.
  \item Use \structure{reusable workflows} (\href{https://docs.github.com/en/actions/using-workflows/reusing-workflows}{docs}): \\
    \url{https://github.com/iiasa/actions/tree/main/.github/workflows}
  \item Job examples: run tests; enforce code style.
  \item Check that Python packages can be built without error.
\end{itemize}

\medskip
Practices:
\begin{itemize}
  \item Run on several events:
    \begin{itemize}
      \item Pushes to pull request branches → check before merging code changes.
      \item Schedule (e.g. nightly) → long-running (‘integration’) tests.
    \end{itemize}
  \item Outcomes monitored → immediately spot disturbances due to changes in ‘upstream’ code.
  \item Relieve users from running tests locally.
\end{itemize}

\end{frame}

\subsection{Test}
\begin{frame}
\frametitle{Test}

Pytest suite in each stack level.
\begin{itemize}
  \item e.g. for \href{https://github.com/iiasa/message-ix-models/tree/main/message_ix_models/tests}{\ttfamily message-ix-models}
\end{itemize}

\bigskip
Practices:
\begin{itemize}
  \item Create and use test \structure{fixtures}.
    \begin{itemize}
      \item Prepared objects, data, etc. that can be used by other tests.
      \item Reduces boilerplate → tests are shorter and easier to read.
    \end{itemize}
  \item Tests operate on \structure{temporary} directories, databases.
  \item Large number of atomic tests that run quickly (for fast CI feedback).
  \item Always add tests on bugs/‘accidental’ breakage of downstream expectations.
\end{itemize}

\end{frame}

\subsection{Adopt a code style}
\begin{frame}
\frametitle{Adopt a code style}

\begin{itemize}
  \item Described \href{https://docs.messageix.org/en/stable/contributing.html\#code-style}{in the documentation itself}.
  \item For Python: \texttt{black}, \texttt{isort}, \texttt{mypy} (type checking), coverage (via Codecov.io).
  \item Team support: plug-ins for different IDEs; education for newcomers about purpose/use of code quality tools.
\end{itemize}

\medskip
Benefits:
\begin{itemize}
  \item “Transparent” and “idiomatic” code: easier to read, review, debug, extend.
\end{itemize}

\end{frame}

\subsection{Document}

\begin{frame}
\frametitle{Document}

\begin{itemize}
  \item \href{https://docs.messageix.org}{docs.messageix.org}: entry point
  \item Cross-link docs for other packages in the stack:
    \begin{itemize}
      \item \href{https://docs.messageix.org/ixmp}{docs.messageix.org/ixmp}
      \item \href{https://docs.messageix.org/models}{docs.messageix.org/models}
      \item \href{https://docs.messageix.org/models-internal}{docs.messageix.org/models-internal} (private, access-controlled)
    \end{itemize}
\end{itemize}

\medskip
Practices:
\begin{itemize}
  \item Documentation modified with changes to the code → PR requirement.
  \item Rebuilt automatically, as soon as changes are merged.
  \item Multiple versions available.
  \item Consider user entry-points, way-finding.
\end{itemize}

\end{frame}

\subsection{Manage user/community interaction}
\begin{frame}
\frametitle{Manage user/community interaction}

\begin{itemize}
  \item Inside: Slack.
  \item Outside: \url{https://github.com/iiasa/message_ix/discussions}.
  \item Google group (deprecated).
\end{itemize}

\medskip
Practices:
\begin{itemize}
  \item Explicit statement of levels of available support.
    \begin{itemize}
      \item Useful both internally and externally.
    \end{itemize}
  \item Duty of responding to outside users distributed across team.
  \item A docs page on \href{https://docs.messageix.org/en/stable/prereqs.html}{pre-requisite knowledge \& skills} → allow users to identify fitness-for-use.
  \item Periodic events:
    \begin{itemize}
      \item MESSAGE\emph{ix} training workshops (next: -tbd- June 2023).
      \item MESSAGE\emph{ix} Community Meeting (next: 24–25 May 2023).
    \end{itemize}
\end{itemize}

\end{frame}

\subsection{Track issues, projects \& work}
\begin{frame}
\frametitle{Track issues, projects \& work}

\begin{itemize}
  \item on GitHub, e.g. \href{https://github.com/iiasa/message_ix/issues}{for MESSAGE\emph{ix}}.
  \item Projects features \url{https://github.com/orgs/iiasa/projects/10}.
\end{itemize}

\medskip
Practices:
\begin{itemize}
  \item Move issues between repos.
  \item Convert “not-really” issues to discussion items.
  \item \structure{Not} a full “agile methodology” as practiced in software industry…
    \begin{itemize}
      \item …selective use of tactics and perspectives.
    \end{itemize}
  \item Explicitly identify who \structure{can} (skills + availability) review code.
    \begin{itemize}
      \item Sometimes pragmatically postpone or skip review \& requirements.
    \end{itemize}
\end{itemize}

\end{frame}

\subsection{Version, schedule releases, deprecate}
\begin{frame}
\frametitle{Version, schedule releases, deprecate}

\medskip
Practices:
\begin{itemize}
  \item Semantic versioning: \url{https://semver.org}
  \item Deprecation policy: notify, wait, remove:
    \begin{itemize}
      \item Follow upstream examples—e.g. \href{https://pandas.pydata.org/docs/development/policies.html\#policies-version}{Pandas}, \href{https://numpy.org/neps/nep-0023-backwards-compatibility.html}{NumPy}—but lighter weight according to available resources.
    \end{itemize}
  \item Release milestones (e.g. \href{https://github.com/iiasa/message_ix/milestones?state=closed}{\ttfamily message\_ix}) for scheduling issues and PRs.
\end{itemize}

\end{frame}

\end{document}
